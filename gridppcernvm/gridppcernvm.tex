%%%%%%%%%%%%%%%%%%%%%%%%%%%%%%%%%%%%%%%%%%%%%%%%%%%%%%%%%%%%%%%%%%%%%%%%%%%%%%%
\section{Creating a GridPP CernVM}
\label{sec:gridppcernvm}
%%%%%%%%%%%%%%%%%%%%%%%%%%%%%%%%%%%%%%%%%%%%%%%%%%%%%%%%%%%%%%%%%%%%%%%%%%%%%%%
If you don't have an account on a grid-ready cluster, don't worry - this
section will show you how to create a \textbf{Virtual Machine} (VM) that
will, essentially, act as a grid User Interface (UI) within whatever
operating system you happen to be using at the moment. We will do this
using the CernVM service~\cite{CernVM2015},
and create a guest CernVM on your host system.
There are a number of reasons to do this:

\begin{enumerate}
\def\labelenumi{\arabic{enumi}.}
\tightlist
\item
  The CernVM can act as a pre-built Grid User Interface (UI) that will
  give you all the tools you need (e.g.~a command line terminal, text
  editors, etc.) to get the most out of the Grid;
\item
  Your CernVM will also give you instant access to the CernVM-File
  System (CVMFS). A lot of the software we will use to manage our grid
  jobs and data is provided using this, with the huge bonus of \emph{not
  needing any installation by you}.
\item
  Most of the Grid tools we will use are compiled to run on Scientific
  Linux 6; With a CernVM you'll be able to use them out of the box;
\item
  The standard Grid Worker Nodes you'll be using to run your software on
  use SL6 machines. If your code runs on your CernVM, it will run on the
  Grid;
\item
  What's more, if everyone uses the CernVM as their Grid UI, we at
  GridPP will only have to support one operating system (i.e.~the SL6
  supplied with the CernVM). If we're singing from the same (virtual)
  hymn sheet, we'll be able to recreate your problems and help you solve
  them more easily.
\end{enumerate}

All you need to provide is the RAM and the hard disk space on your local
host machine.

\begin{infobox}{SL6 cluster access?}
\emph{If you already have access to a user account on an SL6 terminal, for
example on a university computing cluster, you can probably skip this
section.}
\end{infobox}

%=============================================================================
\subsection{An overview of the process}
\label{an-overview-of-the-process}
%=============================================================================
To create and configure a CernVM that will meet our needs, you will need
to:

\begin{enumerate}
\def\labelenumi{\arabic{enumi}.}
\tightlist
\item
  Install some virtualisation software on your host machine;
\item
  Download the appropriate CernVM Virtual Machine baseline image from
  the CernVM service;
\item
  Create a new guest CernVM using the CernVM image;
\item
  Configure the guest CernVM so that it has access to the host hard
  disk;
\item
  Configure the CernVM for Grid use by applying the GridPP
  \emph{contextualisation}.
\end{enumerate}

Let's look at each of these stages in a bit more detail.


%=============================================================================
\subsection{The virtualisation software}
\label{the-virtualisation-software}
%=============================================================================
You can find a list of compatible virtualisation software solutions on
the CernVM image download page
\href{http://cernvm.cern.ch/portal/downloads}{here}. It doesn't really
matter which you use but we've had success with
\href{http://download.virtualbox.org/virtualbox/4.3.12/VirtualBox-4.3.12-93733-Win.exe}{this
version of VirtualBox} (Windows 7).

\begin{warningbox}{Setting up the virtualisation software}
\emph{This is the one bit that's a bit tricky for us to support - how you do
this will depend on your local machine and its setup. Remember, search
engines and StackOverflow are your friends here.}
\end{warningbox}

%=============================================================================
\subsection{Creating your CernVM}
\label{creating-your-cernvm}
%=============================================================================
Depending on the virtualisation solution you picked (and got working)
above, download the baseline CernVM image file from here.

Then use your virtualisation software to create a new VM from the
downloaded image. You should be able to find instructions for how to do
this from your virtualisation software provider. Use as much RAM as you
can spare (up to about half of your host machine's total RAM) and use a
virtual hard disk of about 30 GB.

Once you have created your CernVM, but \emph{before you start it}, you
will need setup the \emph{contextualisation} for your CernVM.

\subsection{Contextualising your
CernVM}\label{contextualising-your-cernvm}

The CernVM service offers the ability to contextualise a CernVM with
pre-defined settings, environment variables, etc. that are put in place
when the CernVM is first booted. This is known as \emph{pairing} the VM.
You can create your own contextuallisations, but it is also possible for
individuals to create public contexts (e.g.~for LHC experiments, open
data initiatives, etc.) that anyone can use.

We have created such a context for GridPP. You can use it to get going
with the Grid. Importantly, you do not need a CERN account to do this,
so it is possible for anyone with a grid certificate to use it!

To pair your local CernVM with the GridPP context:

\begin{enumerate}
\def\labelenumi{\arabic{enumi}.}
\tightlist
\item
  \textbf{Log in to the CernVM service}: You can do this
  \href{https://cernvm-online.cern.ch/user/login}{here}. If you don't
  have a CERN account, you can register
  \href{https://cernvm-online.cern.ch/user/register}{here}.
\item
  \textbf{Visit the CernVM Marketplace}: This can be found
  \href{https://cernvm-online.cern.ch/market/list}{here}.
\item
  \textbf{Select the GridPP CernVM context}: Select \emph{Experimental}
  from the panel on the right, and then click on the
  \texttt{gridpp-cernvm} context.
\item
  \textbf{Boot up your local CernVM}: Once the boot process has
  finished, you should be presented with a login prompt. Don't enter
  anything yet.
\item
  \textbf{Pair with the GridPP context}: Returning to the GridPP CernVM
  Marketplace, click on the \emph{Pair} button on the panel on the
  right. This will generate a six-figure PIN.
\item
  \textbf{Apply the context to your VM}: Returning to your booted-up
  CernVM, enter the PIN (preceeded by a hash, as instructed at the login
  prompt). The CernVM webpage will now update indicating the VM has been
  successfully paired. Your VM will then be restarted and
  contextualised.
\item
  \textbf{Log in to your GridPP CernVM}: Once the context has been
  applied to your CernVM, you will be presented with a graphical login
  screen (as opposed to the text login prompt). Use the username
  \texttt{gridpp} and password \texttt{gridpp} to log in to your CernVM.
\end{enumerate}

And that's it! You now have a shiny new GridPP CernVM from which you'll
be able to access the Grid.

%=============================================================================
\subsection{Two final things}
\label{two-final-things}
%=============================================================================

\subsubsection{Configure git}\label{configure-git}

We'll be using \texttt{git} (and GitHub) to access various pieces of
code and scripts. The CernVM comes with \texttt{git} installed but
you'll need to configure it with your username and email address. This
can be done with the following commands:

\begin{Shaded}
\begin{Highlighting}[]
\NormalTok{$ }\KeywordTok{git} \NormalTok{config --global user.name }\StringTok{"Ada Lovelace"}
\NormalTok{$ }\KeywordTok{git} \NormalTok{config --global user.email alovelace@qmul.ac.uk}
\end{Highlighting}
\end{Shaded}

\subsubsection{Local hard disk access}\label{local-hard-disk-access}

Before we move on, it's important to note that you will need to be able
to access the hard disk of your host machine from that of your guest
CernVM. This is so that you can move any files that you need across to
the CernVM - most importantly, your Grid certificate file.

With VirtualBox, for example, this is achieved using the Shared Folders
functionality.

Before proceeding, you should make sure that you can access the parts of
your local hard disk that contain any software or data that you want to
copy across.

%=============================================================================
\subsection{Checklist}
\label{creating-a-grid-ui-with-a-gridpp-cernvm---checklist}
%=============================================================================

\begin{itemize}
\tightlist
\item
  I have installed some virtualisation software on my local machine that
  is compatible with the CernVM images listed at:

  \url{http://cernvm.cern.ch/portal/downloads}.
\item
  I have downloaded the corresponding CernVM image to my local machine
  from the CernVM image download page;
\item
  I have created a new guest VM on my local machine using the CernVM
  image;
\item
  I have registered with the CernVM service
  (or I am a CERN computing account holder);
\item
  I have generated a six-digit PIN to pair my CernVM with the GridPP
  context via the CernVM Marketplace;
\item
  I have entered the six-digit PIN (preceeded by the hash symbol) into
  the login prompt of my newly-booted CernVM;
\item
  I have logged in to my new GridPP CernVM using the \texttt{gridpp}
  user account;
\item
  I have accessed folders from my local machine's hard disk from my
  GridPP CernVM.
\end{itemize}


%=============================================================================
\subsection{Testing}
\label{creating-a-grid-ui-with-a-gridpp-cernvm---testing}
%=============================================================================

\begin{itemize}
\item
  \textbf{Downloading the CernVM image}: Your \texttt{Downloads} folder
  (or whichever location you chose to download the CernVM image file)
  contains the image file.

\begin{Shaded}
\begin{Highlighting}[]
\NormalTok{$ }\KeywordTok{cd} \NormalTok{~/Downloads}
\NormalTok{$ }\KeywordTok{ls} \NormalTok{-l }\KeywordTok{|} \KeywordTok{grep} \NormalTok{cernvm}
\KeywordTok{cernvm-3.5.1.iso}
\end{Highlighting}
\end{Shaded}
\item
  \textbf{Creating the CernVM}: When you start up your CernVM from your
  virtualisation software, you are eventually presented with a login
  screen.
\end{itemize}

\begin{Shaded}
\begin{Highlighting}[]
\KeywordTok{Welcome} \NormalTok{to CERN Virtual Machine, version 3.5.1.4}
  \KeywordTok{based} \NormalTok{on Scientific Linux release 6.6 (Carbon)}
  \KeywordTok{Kernel} \NormalTok{3.18.20-18.cernvm.x86_64 on an x86_64}

\KeywordTok{IP} \NormalTok{Address of this VM: [IP address]}
\KeywordTok{In} \NormalTok{order to apply cernvm-online context, use }\CommentTok{#<PIN> as user name.}

\KeywordTok{localhost} \NormalTok{login: _}
\end{Highlighting}
\end{Shaded}

\begin{itemize}
\item
  \textbf{Registering with the CernVM Service}: You can login via
  \href{https://cernvm-online.cern.ch/user/login}{this page}. When you
  access \href{https://cernvm-online.cern.ch/}{this page} you are
  redirected to the
  \href{https://cernvm-online.cern.ch/dashboard}{CernVM Dashboard}.
\item
  \textbf{Pairing your CernVM with the GridPP CernVM context}: After
  selecting the GridPP CernVM context in the
  \href{https://cernvm-online.cern.ch/market/list}{CernVM Marketplace}
  and clicking on the \emph{Pair} button on the right-hand panel, you
  are presented with a six-digit number. After entering the PIN into
  your CernVM's login screen (preceeded by the hash symbol), your CernVM
  reboots to display a graphical CernVM login screen. Meanwhile, the
  CernVM Pairing web page has updated to display the message,
  \textgreater{} \textbf{Setup instance - Completed} \textgreater{}
  \textgreater{} Your vm is now paired with CernVM Online! Upon logging
  in to your new CernVM with the username \texttt{gridpp} and the
  password \texttt{gridpp}, you are presented with the CernVM desktop.
\item
  \textbf{Accessing your local machine's hard disk from you CernVM}: You
  can access (from the command line or otherwise) folders on your local
  machine's hard disk on the GridPP CernVM. So (for example) if you're
  using VirtualBox with the \emph{Shared Folders} functionality, after
  adding the folders you want access to via the VM's \emph{Settings} and
  rebooting the VM you'll do something like this:

\begin{Shaded}
\begin{Highlighting}[]
\NormalTok{$ }\KeywordTok{sudo} \NormalTok{usermod -a -G vboxsf gridpp}
\NormalTok{[}\KeywordTok{sudo}\NormalTok{] password for gridpp: }\CommentTok{# Enter 'gridpp' here.}
\end{Highlighting}
\end{Shaded}

  before loggin out and logging in again. You'll then be able to access
  the folders you specified.

\begin{Shaded}
\begin{Highlighting}[]
\NormalTok{$ }\KeywordTok{cd} \NormalTok{/media/sf_alovelace/ }\CommentTok{# Shared Folder "alovelace" }
\NormalTok{$ }\KeywordTok{ls}
\KeywordTok{punch-card-01.dat} \NormalTok{dad-poem-001.txt my-poem-005.txt}
\end{Highlighting}
\end{Shaded}
\end{itemize}

