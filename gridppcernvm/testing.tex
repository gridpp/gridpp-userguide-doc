%=============================================================================
\subsection{Testing}
\label{creating-a-grid-ui-with-a-gridpp-cernvm---testing}
%=============================================================================

\begin{itemize}
\item
  \textbf{Downloading the CernVM image}: Your \texttt{Downloads} folder
  (or whichever location you chose to download the CernVM image file)
  contains the image file.

\begin{Shaded}
\begin{Highlighting}[]
\NormalTok{$ }\KeywordTok{cd} \NormalTok{~/Downloads}
\NormalTok{$ }\KeywordTok{ls} \NormalTok{-l }\KeywordTok{|} \KeywordTok{grep} \NormalTok{cernvm}
\KeywordTok{cernvm-3.5.1.iso}
\end{Highlighting}
\end{Shaded}
\item
  \textbf{Creating the CernVM}: When you start up your CernVM from your
  virtualisation software, you are eventually presented with a login
  screen.
\end{itemize}

\begin{Shaded}
\begin{Highlighting}[]
\KeywordTok{Welcome} \NormalTok{to CERN Virtual Machine, version 3.5.1.4}
  \KeywordTok{based} \NormalTok{on Scientific Linux release 6.6 (Carbon)}
  \KeywordTok{Kernel} \NormalTok{3.18.20-18.cernvm.x86_64 on an x86_64}

\KeywordTok{IP} \NormalTok{Address of this VM: [IP address]}
\KeywordTok{In} \NormalTok{order to apply cernvm-online context, use }\CommentTok{#<PIN> as user name.}

\KeywordTok{localhost} \NormalTok{login: _}
\end{Highlighting}
\end{Shaded}

\begin{itemize}
\item
  \textbf{Registering with the CernVM Service}: You can login via
  \href{https://cernvm-online.cern.ch/user/login}{this page}. When you
  access \href{https://cernvm-online.cern.ch/}{this page} you are
  redirected to the
  \href{https://cernvm-online.cern.ch/dashboard}{CernVM Dashboard}.
\item
  \textbf{Pairing your CernVM with the GridPP CernVM context}: After
  selecting the GridPP CernVM context in the
  \href{https://cernvm-online.cern.ch/market/list}{CernVM Marketplace}
  and clicking on the \emph{Pair} button on the right-hand panel, you
  are presented with a six-digit number. After entering the PIN into
  your CernVM's login screen (preceeded by the hash symbol), your CernVM
  reboots to display a graphical CernVM login screen. Meanwhile, the
  CernVM Pairing web page has updated to display the message,
  \textgreater{} \textbf{Setup instance - Completed} \textgreater{}
  \textgreater{} Your vm is now paired with CernVM Online! Upon logging
  in to your new CernVM with the username \texttt{gridpp} and the
  password \texttt{gridpp}, you are presented with the CernVM desktop.
\item
  \textbf{Accessing your local machine's hard disk from you CernVM}: You
  can access (from the command line or otherwise) folders on your local
  machine's hard disk on the GridPP CernVM. So (for example) if you're
  using VirtualBox with the \emph{Shared Folders} functionality, after
  adding the folders you want access to via the VM's \emph{Settings} and
  rebooting the VM you'll do something like this:

\begin{Shaded}
\begin{Highlighting}[]
\NormalTok{$ }\KeywordTok{sudo} \NormalTok{usermod -a -G vboxsf gridpp}
\NormalTok{[}\KeywordTok{sudo}\NormalTok{] password for gridpp: }\CommentTok{# Enter 'gridpp' here.}
\end{Highlighting}
\end{Shaded}

  before loggin out and logging in again. You'll then be able to access
  the folders you specified.

\begin{Shaded}
\begin{Highlighting}[]
\NormalTok{$ }\KeywordTok{cd} \NormalTok{/media/sf_alovelace/ }\CommentTok{# Shared Folder "alovelace" }
\NormalTok{$ }\KeywordTok{ls}
\KeywordTok{punch-card-01.dat} \NormalTok{dad-poem-001.txt my-poem-005.txt}
\end{Highlighting}
\end{Shaded}
\end{itemize}
