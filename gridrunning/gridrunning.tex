%%%%%%%%%%%%%%%%%%%%%%%%%%%%%%%%%%%%%%%%%%%%%%%%%%%%%%%%%%%%%%%%%%%%%%%%%%%%%%%
\section{Moving Your Workflow to the Grid}
\label{sec:gridrunning}
%%%%%%%%%%%%%%%%%%%%%%%%%%%%%%%%%%%%%%%%%%%%%%%%%%%%%%%%%%%%%%%%%%%%%%%%%%%%%%%
Now that you have your Grid certificate, and it is installed in your
browser and in your \texttt{\textasciitilde{}/.globus} directory, you're
ready to try submitting a job to the Grid with DIRAC.

%=============================================================================
\subsection{Activate DIRAC!}
\label{activate-dirac}
%=============================================================================
Thanks to CVMFS and the \href{https://github.com/ganga-devs/}{Ganga
team}, activating the GridPP DIRAC functionality is easy:

\begin{Shaded}
\begin{Highlighting}[]
\NormalTok{$ }\KeywordTok{source} \NormalTok{/cvmfs/ganga.cern.ch/dirac_ui/bashrc}
\end{Highlighting}
\end{Shaded}

You should now be able to tab-complete \texttt{dirac-} to see all of the
DIRAC commands that are available.

\begin{infobox}{DIRAC on CVMFS}
\emph{The inclusion of the pre-configured DIRAC UI in the Ganga CVMFS
repository means that you no longer need to install your own DIRAC UI.
Which makes life a lot easier}\ldots{}
\end{infobox}

%-----------------------------------------------------------------------------
\subsubsection{Generate a Grid proxy}
\label{generate-a-grid-proxy}
%-----------------------------------------------------------------------------
With DIRAC activated via CVMFS, you should now be able generate a
DIRAC-specific proxy to be used as a member of a DIRAC-enabled VO. Your
proxy is a file that identifies you on the Grid, letting the system know
that you're good for using its resources. If, for example, you were a
member of the \texttt{gridpp} catch-all VO you would use the following
commands to generate a proxy as a \texttt{gridpp\_user}:

\begin{Shaded}
\begin{Highlighting}[]
\NormalTok{$ }\KeywordTok{dirac-proxy-init} \NormalTok{-g gridpp_user -M}
\KeywordTok{Generating} \NormalTok{proxy... }
\KeywordTok{Enter} \NormalTok{Certificate password:}
\KeywordTok{Added} \NormalTok{VOMS attribute /gridpp }
\KeywordTok{Uploading} \NormalTok{proxy for gridpp_user... }
\KeywordTok{Proxy} \NormalTok{generated: }
\KeywordTok{subject}      \NormalTok{: /C=UK/O=eScience/OU=QueenMaryLondon/L=Computing/CN=ada lovelace/CN=proxy}
\KeywordTok{issuer}       \NormalTok{: /C=UK/O=eScience/OU=QueenMaryLondon/L=Computing/CN=ada lovelace}
\KeywordTok{identity}     \NormalTok{: /C=UK/O=eScience/OU=QueenMaryLondon/L=Computing/CN=ada lovelace}
\KeywordTok{timeleft}     \NormalTok{: 23:53:59}
\KeywordTok{DIRAC} \NormalTok{group  : gridpp_user}
\KeywordTok{path}         \NormalTok{: /tmp/x509up_u501}
\KeywordTok{username}     \NormalTok{: ada.lovelace}
\KeywordTok{properties}   \NormalTok{: NormalUser}
\KeywordTok{VOMS}         \NormalTok{: True}
\KeywordTok{VOMS} \NormalTok{fqan    : [}\StringTok{'/gridpp'}\NormalTok{] }
\end{Highlighting}
\end{Shaded}

If you get output that looks like the above, then it has all worked.

\begin{infobox}{Proxy status}
\emph{You can check the status of your DIRAC proxy at any time with the
dirac-proxy-info command.}
\end{infobox}

%-----------------------------------------------------------------------------
\subsubsection{Configuring Ganga to use the DIRAC backend}
\label{configuring-ganga-to-use-the-dirac-backend}
%-----------------------------------------------------------------------------
There's just one more step required before you can enjoy Grid running
with Ganga - you need to add the following to your
\texttt{\textasciitilde{}/.gangarc} configuration file:

\begin{Shaded}
\begin{Highlighting}[]
\NormalTok{[}\KeywordTok{Configuration}\NormalTok{]}
\KeywordTok{RUNTIME_PATH} \NormalTok{= GangaDirac}

\NormalTok{[}\KeywordTok{LCG}\NormalTok{]}
\KeywordTok{GLITE_SETUP} \NormalTok{= /cvmfs/ganga.cern.ch/dirac_ui/bashrc}

\NormalTok{[}\KeywordTok{DIRAC}\NormalTok{]}
\KeywordTok{DiracEnvSource} \NormalTok{= /cvmfs/ganga.cern.ch/dirac_ui/bashrc}

\NormalTok{[}\KeywordTok{defaults_DiracProxy}\NormalTok{]}
\KeywordTok{group} \NormalTok{= }\KeywordTok{<}\NormalTok{dirac user group}\KeywordTok{>}

\NormalTok{[}\KeywordTok{defaults_DiracFile}\NormalTok{]}
\KeywordTok{defaultSE} \NormalTok{= }\KeywordTok{<}\NormalTok{your SE of choice}\KeywordTok{>}
\end{Highlighting}
\end{Shaded}

where \texttt{\textless{}dirac\ user\ group\textgreater{}} should be
replaced by your default VO (e.g. \texttt{gridpp\_user}) and
\texttt{\textless{}your\ SE\ of\ choice\textgreater{}} should be
replaced by a suitable Storage Element, e.g.
\texttt{UKI-LT2-QMUL2-disk}.

\begin{infobox}{Finding an SE}
\emph{You can find a list of Storage Elements names by using the
dirac-dms-show-se-status command from the command line.}
\end{infobox}

You can then re-start Ganga; it will now be ready to connect to the
DIRAC backend.

%=============================================================================
\subsection{Run your job on the Grid}
\label{run-your-job-on-the-grid}
%=============================================================================
Now you're ready to run your job on the Grid. First, make a copy of the
\texttt{local\_job.py} script:

\begin{Shaded}
\begin{Highlighting}[]
\NormalTok{$ }\KeywordTok{cd} \OtherTok{$WORKINGDIR}
\NormalTok{$ }\KeywordTok{cp} \NormalTok{local_job.py dirac_job.py}
\NormalTok{$ }\KeywordTok{vim} \NormalTok{dirac_job.py}
\end{Highlighting}
\end{Shaded}

All you need to do is add the line \texttt{j.backend=Dirac()} before
submitting. That's it. That's all there is to it.

\begin{Shaded}
\begin{Highlighting}[]
\NormalTok{$ }\KeywordTok{cat} \NormalTok{dirac_job.py}
\KeywordTok{j} \NormalTok{= Job()}
\KeywordTok{j.name} \NormalTok{= }\StringTok{"CERN@school_dirac_01"}
\KeywordTok{j.application} \NormalTok{= Executable()}
\KeywordTok{j.application.exe} \NormalTok{= File(}\StringTok{'run.sh'}\NormalTok{)}
\KeywordTok{j.application.args} \NormalTok{= [}\StringTok{'B06-W0212/2014-04-02-150255/'}\NormalTok{]}
\KeywordTok{j.inputfiles} \NormalTok{= [ LocalFile(}\StringTok{'CERNatschool_backgroundrad_dataset.zip'}\NormalTok{) ]}
\KeywordTok{j.outputfiles} \NormalTok{= [ LocalFile(}\StringTok{'frames.json'}\NormalTok{), }\KeywordTok{LocalFile}\NormalTok{(}\StringTok{'log_process-frames.log'}\NormalTok{), }
\KeywordTok{LocalFile}\NormalTok{(}\StringTok{'output_images.tar'}\NormalTok{) ]}
\KeywordTok{j.backend} \NormalTok{= Dirac()}
\KeywordTok{j.submit}\NormalTok{()}
\end{Highlighting}
\end{Shaded}

(\emph{Oh, you may want to change the job name too}.)

If all has gone to plan, not only will you now be able to monitor your
job via your local Ganga instance (i.e.~with the \texttt{jobs} command),
you can see it on the \href{http://dirac.gridpp.ac.uk}{GridPP DIRAC web
portal}. Select \emph{Jobs} from the top-left menu below the URL bar,
then \emph{Job Monitor}.

\begin{warningbox}{Submission time}
\emph{Job submission to the Grid is not an instant process - a bit annoying
when you're submitting one or two test jobs (which is why local testing
wit Ganga is great!), but not such an issue with thousands of jobs. You
may wish to make a cup of tea, or do a bit of washing up.}
\end{warningbox}

Once the job shows up as green in the web portal, it's completed and you
can retrieve the output exactly as before.

So there you go - your first \emph{bona fida} grid job. Note that:

\begin{itemize}
\tightlist
\item Once your Grid stuff/DIRAC configuration was done, all it took to switch
to Grid running was a single line of code.
\item Thanks to CVMFS, you didn't have to do anything extra to deploy
your software to the Grid;
\item You didn't need to care about where the job actually ran - Ganga and DIRAC
sorted that all for you.
\end{itemize}

There's only one more thing to look at now before we've covered all of
the Grid-bases - getting data on and off the Grid. We'll look at that in
the next section.

\newpage

%=============================================================================
\subsection{Checklist}
\label{moving-the-example-workflow-to-the-grid---checklist}
%=============================================================================

\begin{itemize}
\tightlist
\item
  I can activate DIRAC by sourcing the DIRAC \texttt{bashrc} script from
  CVMFS;
\item
  I can generate a Grid proxy using the \texttt{dirac-proxy-init}
  command;
\item
  I can submit the CERN@school example workflow job to the Grid using
  Ganga;
\item
  I can monitor my Grid job(s) using the
  \href{https://dirac.gridpp.ac.uk}{GridPP DIRAC web portal}.
\end{itemize}


%=============================================================================
\subsection{Testing}
\label{moving-the-example-workflow-to-the-grid---testing}
%=============================================================================

\begin{itemize}
\item
  \textbf{Sourcing the DIRAC environment}: You can test the DIRAC
  environment variables have been set using the \texttt{echo} command.
  For example:

\begin{Shaded}
\begin{Highlighting}[]
\NormalTok{$ }\KeywordTok{echo} \OtherTok{$DIRAC}
\KeywordTok{/cvmfs/ganga.cern.ch/dirac_ui/}
\end{Highlighting}
\end{Shaded}

  If the DIRAC home directory on CVMFS is not listed, the environment
  variables have not been set correctly.
\item
  \textbf{Generating a DIRAC proxy}: You can test if your proxy
  generation has been successful by using the \texttt{dirac-proxy-info}
  command.
\item
  \textbf{Successful running of the CERN@school example job}: As with
  the locally-run example, once you have run and retrieved the images
  from the example job the first frame image should look something like
  that seen in Figure~\ref{fig:localrunningframe}.
\end{itemize}

