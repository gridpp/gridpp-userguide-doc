%=============================================================================
\subsection{Preparing your Grid certificate}
\label{sec:certprep}
%=============================================================================
Ganga will assume that your grid certificate is in a certain location
and in a certain format in order to use it. Your grid certificate
therefore needs to be moved and prepared accordingly - which you can do
by following the instructions below.

%-----------------------------------------------------------------------------
\subsubsection{Moving your Grid certificate to your UI}
\label{moving-your-grid-certificate-to-your-ui}
%-----------------------------------------------------------------------------
The first thing to do is move your Grid certificate (the one you got
after following the instructions in Section~\ref{sec:gridcert})
%\href{../getting-on-the-grid/grid-certificate.md}{here}
to the \texttt{\textasciitilde{}/.globus/} directory in your home folder.

\begin{Shaded}
\begin{Highlighting}[]
\NormalTok{$ }\KeywordTok{cd} \NormalTok{~}
\NormalTok{$ }\KeywordTok{pwd}
\NormalTok{[}\KeywordTok{Your} \NormalTok{home directory.]}
\NormalTok{$ }\KeywordTok{mkdir} \NormalTok{.globus}
\NormalTok{$ }\KeywordTok{cp} \NormalTok{[The location of your certificate file.]/[Your certificate filename].p12 ./.globus/.}
\end{Highlighting}
\end{Shaded}

\begin{warningbox}{Certificates on CernVMs}
\emph{If you are using a CernVM and have moved your personal Grid certificate
file to it, you should change the password of the gridpp account so that
no-one else can use it. This can be done in the standard UNIX way with
the passwd command.}
\end{warningbox}

%-----------------------------------------------------------------------------
\subsubsection{Converting your Grid certificate}
\label{converting-your-grid-certificate}
%-----------------------------------------------------------------------------
In order to use your Grid certificate, you need to convert them into
separate certificate and key files. Don't worry, this straightforward
enough to do with the following commands:

\begin{Shaded}
\begin{Highlighting}[]
\NormalTok{$ }\KeywordTok{cd} \NormalTok{~/.globus}
\NormalTok{$ }\KeywordTok{openssl} \NormalTok{pkcs12 -in [Your certificate filename.].p12 -clcerts -nokeys -out usercert.pem}
\KeywordTok{Enter} \NormalTok{Import Password:}
\KeywordTok{MAC} \NormalTok{verified OK}
\NormalTok{$ }\KeywordTok{openssl} \NormalTok{pkcs12 -in [Your certificate filename.].p12 -nocerts -out userkey.pem}
\KeywordTok{Enter} \NormalTok{Import Password:}
\KeywordTok{MAC} \NormalTok{verified OK}
\KeywordTok{Enter} \NormalTok{PEM pass phrase:}
\KeywordTok{Verifying} \NormalTok{- Enter PEM pass phrase:}
\end{Highlighting}
\end{Shaded}

You will then need to change the file permission settings on the two
newly-generated files:

\begin{Shaded}
\begin{Highlighting}[]
\NormalTok{$ }\KeywordTok{chmod} \NormalTok{400 userkey.pem}
\NormalTok{$ }\KeywordTok{chmod} \NormalTok{600 usercert.pem}
\end{Highlighting}
\end{Shaded}

And that's it! You're now ready to use GridPP DIRAC with Ganga. It may
have seemed like a lot of work, but hopefully the next section
%\href{../example-workflow-grid/example-workflow-grid.md}{next section}
will demonstrate it will all have been worth it.
