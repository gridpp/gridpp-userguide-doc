%%%%%%%%%%%%%%%%%%%%%%%%%%%%%%%%%%%%%%%%%%%%%%%%%%%%%%%%%%%%%%%%%%%%%%%%%%%%%%%
\section{Getting on the Grid}
\label{sec:onthegrid}
%%%%%%%%%%%%%%%%%%%%%%%%%%%%%%%%%%%%%%%%%%%%%%%%%%%%%%%%%%%%%%%%%%%%%%%%%%%%%%%
We have run local jobs so far. Now it's time to get on the grid - after
which, we will be able to configure Ganga to submit our jobs to the
GridPP DIRAC system and so access all of the computing and data
resources GridPP has to offer. The following sections will cover:

\begin{itemize}
\tightlist
\item
  Getting a grid certificate (Section~\ref{sec:gridcert});
\item
  Joining a Virtual Organisation (Section~\ref{sec:joinvo});
\item
  Logging on to GridPP DIRAC.
\end{itemize}

We'll find out more about what each of these steps mean and entail as we
go.

\begin{warningbox}{Thinking ahead}
\emph{Some of these steps require interaction with a human being. For example,
to get a grid certificate you have to visit your local Registration
Authority in person with photographic ID. Please bear this in mind
before you begin - it is not an automated process and may take a little
time. It will be worth it though!}
\end{warningbox}

Let's start by getting you a Grid certificate, shall we?

%=============================================================================
\subsection{Your Grid certificate}
\label{sec:gridcert}
%=============================================================================
Your grid certificate is your passport to the grid. It will give you
access to the vast array of computational resources that GridPP (and the
wLCG) has to offer. As such, getting a grid certificate is an
understandably non-trivial, multi-step process. For example, you will
need to identify yourself to your local Registration Authority (RA) so
that the grid knows who you are.

In concrete terms, a grid certificate is a .p12 file (i.e.~a pkcs12 web
browser certificate file) that you will later convert into a \emph{user
key} file and a \emph{user certificate} file. Encoded according to the
X.509 standard, these are used by the grid to confirm who are you are
and so give you access to grid resources.

%-----------------------------------------------------------------------------
\subsubsection{Requesting a Grid certificate}
\label{requesting-a-grid-certificate}
%-----------------------------------------------------------------------------
Grid certificates in the UK are managed by the
UK e-Science Certificate Authority\footnote{%
See \href{http://ngs.ac.uk/ukca}{http://ngs.ac.uk/ukca}}.
To start the process, you need to choose a web browser that you will have
consistent access to. We recommend Firefox as this process has been
tested and confirmed to work with Firefox on most Operating Systems.
This is because you need to use the same system for both requesting your
certificate and retrieving it when it is ready.

\begin{warningbox}{Temporary logins}
\emph{Do not use a temporary login anywhere when requesting a grid
certificate.}
\end{warningbox}

Using your browser of choice visit the CA portal\footnote{%
See \href{https://portal.ca.grid-support.ac.uk/caportal/}{https://portal.ca.grid-support.ac.uk/caportal/}}
and select the \emph{Request New User Certificate} option. This almost goes
without saying, but make sure you supply a \textbf{valid email address}
which you can access. You will also be asked to do things like supply a
PIN and passwords that you will need later on, so \textbf{make sure you
write everything down}!

\begin{infobox}{Registration Authroties}
\emph{You will need to select a Registration Authority (RA) as part of this
process. If your institution does not have its own RA, select the
nearest on the drop-down menu. You will need to visit the RA in person
with some photographic identification, so don't pick one that is too far
away! If no contact information is listed for a given RA, they will
almost certainly be retrievable using a Search Engine of Your Choice or
via their department's webpage. They will be delighted to hear from you!}
\end{infobox}

Further instructions will then be emailed to you at the email address
you have supplied during the registration process. Once that has
happened you should get a further email from someone at the RA asking
you to visit them in person to complete the validation process.

\begin{warningbox}{Who are you?}
\emph{You may also be asked to supply a letter of recommendation (or, rather,
an email from a suitable authority) explaining why you need to use the
grid and with whom you will be working. If you are unsure about who to
ask for this, please contact us\footnote{%
See \href{https://www.gridpp.ac.uk/contact}{https://www.gridpp.ac.uk/contact}}
and we should be able to help you out.}
\end{warningbox}

%-----------------------------------------------------------------------------
\subsubsection{Installing your Grid certificate in your web browser}
\label{installing-your-grid-certificate-in-your-web-browser}
%-----------------------------------------------------------------------------
Assuming all has gone to plan, you should receive a confirmation email
with a link that will let you download your grid certificate file and
install it in your browser. You will now be able to export and backup
your grid certificate using your browser's certificate management
functionality. This process will vary from browser to browser and from
OS to OS, so consult the
\href{http://www.ngs.ac.uk/ukca/certificates}{UK CA documentation} if in
doubt.

Congratulations - now you can be identified on the grid, you're ready to
join a \term{Virtual Organisation}.


%=============================================================================
\subsection{Joining a Virtual Organisation}
\label{sec:joinvo}
%=============================================================================
Your Grid certificate identifies you to the grid as an individual user,
but it's not enough on its own to allow you to use grid resources; you
also need to join a \term{Virtual Organisation} (VO). These are essentially
just user groups - typically one per experiment - and individual
Resource Centres (RCs) can choose to support work by users of a
particular VO. Most RCs support the four VOs associated with the Large
Hadron Collider (LHC) experiments. The sign-up procedure varies from VO
to VO. UK-based VOs typically require a manual approval step, while LHC
VOs require an active CERN account. If you are already part of an
experiment that is represented by a VO, they should provide you with any
specific instructions you need to join.

If you're interesting in using the grid but are not (yet) working as
part of a user community already represented by a VO, worry not. GridPP
have created a catch-all VO -
\href{https://voms.gridpp.ac.uk:8443/voms/gridpp/}{\texttt{gridpp}} -
and four Regional Virtual Organisations (RVOs) corresponding to the four
Tier 2s that can be joined to test out what the grid has to offer. Once
you have used these ``incubator'' VOs to see if the Grid meets your
needs, you can then think about creating your own Virtual Organisation
to represent your user community.

\begin{infobox}{Is there a VO for you already?}
\emph{Your user community may already have a VO associated with it. Check the
GridPP wiki page of supported VOs to see if you can join that to speed
things up.}
\end{infobox}

%-----------------------------------------------------------------------------
\subsubsection{Joining an incubator VO}
\label{joining-an-incubator-vo}
%-----------------------------------------------------------------------------
\begin{infobox}{Just browsing}
\emph{Some users have reported that the VOMS registration described below
fails using the Safari web browser. We have tried and tested the process
using Mozilla Firefox.}
\end{infobox}

\begin{infobox}{Trusting the VOMS servers}
\emph{Please ignore any ``untrusted connection'' warnings when accessing the
VOMS server pages. GridPP is aware that the VOMS server uses unsigned
certificates, but this situation is unlikely to be resolved any time
soon.}
\end{infobox}

%-----------------------------------------------------------------------------
\subsubsection{Joining the GridPP VO}
\label{joining-the-gridpp-vo}
%-----------------------------------------------------------------------------

To join the
\href{https://voms.gridpp.ac.uk:8443/voms/gridpp}{\texttt{gridpp}} VO,
visit
\href{https://voms.gridpp.ac.uk:8443/voms/gridpp/register/start.action}{this
page} using a browser that has your grid certificate installed and
follow the instructions.

%-----------------------------------------------------------------------------
\subsubsection{Joining a Regional VO}
\label{joining-a-regional-vo}
%-----------------------------------------------------------------------------
Likewise, you can join one of the four regional VOs:

\begin{itemize}
\item
\texttt{vo.londongrid.ac.uk}
\item
\texttt{vo.northgrid.ac.uk}
\item
\texttt{vo.scotgrid.ac.uk}
\item
\texttt{vo.southgrid.ac.uk}
\end{itemize}

\begin{infobox}{Confirming VO your membership}
\emph{Your VO membership request needs to be confirmed manually by one of the
VO administrators, so please wait for the membership confirmation email
to arrive before proceeding. You may wish to keep an eye on your junk
folder(s) too.}
\end{infobox}

Once you have joined a VO, congratulations - you are ready to start
using the Grid!


%=============================================================================
\subsection{Logging on with GridPP DIRAC}
\label{sec:logondirac}
%=============================================================================
There are many ways of accessing and using Grid resources. Larger
organisations - such as the four LHC experiments - have developed their
own frameworks, architectures and mechanisms to enable their members to
run jobs and access experimental data.

One such framework -- \term{DIRAC}~\cite{DIRAC2010} -- is used by
LHCb~\cite{LHCb2008}, but also many other Grid
projects, to manage grid jobs and storage.
You can read more about DIRAC
(Distributed Infrastructure with Remote Agent Control)
on their website\footnote{%
See \href{http://diracgrid.org}{http://diracgrid.org}}
or in~\cite{DIRAC2010},
but for our purposes all you need to
know for now is that DIRAC provides a way for you to access grid
resources without worrying too much about what's going on behind the
scenes.

%-----------------------------------------------------------------------------
\subsubsection{The GridPP DIRAC instance}
\label{the-gridpp-dirac-instance}
%-----------------------------------------------------------------------------
The Imperial College London GridPP Resource Centre (RC) hosts an
instance of DIRAC on behalf of GridPP~\cite{GRIDPPDIRAC2015a,GRIDPPDIRAC2015b}.
The GridPP DIRAC instance is is
capable of serving multiple VOs, providing grid job and data management
capabilities for smaller, non-LHC user communities wishing to make use
of GridPP resources. As a new user, you are automatically registered
with the GridPP DIRAC instance and the Virtual Organisations you have
joined. You can then test out various bits of grid functionality to
determine if grid computing will meet your needs and the needs of your
users.

You can interact with the Grid via the GridPP DIRAC web portal at:

\url{https://dirac.gridpp.ac.uk}

When you access the portal, you should
see yourself listed as a \textbf{Visitor} in drop-down menu in the
bottom-right corner of the browser. Once you have joined one or more
DIRAC-supported Virtual Organisations, you will be able to select which
VO you use DIRAC as using this drop-down menu.

\begin{infobox}{The GridPP DIRAC mailing list}
\emph{You should also join the GridPP DIRAC mailing list to keep informed of
the latest developments and receive notices of any downtime. You can
join the mailing list here.}
\end{infobox}

So you've accessed the GridPP DIRAC web portal. Congratulations!
However, as discussed, we'll be using GridPP DIRAC via the Ganga
interface. In order to do that, you'll need to install your Grid
certificate on your local machine - and the instructions for
doing this are in the next section.


%=============================================================================
\subsection{Preparing your Grid certificate}
\label{sec:certprep}
%=============================================================================
Ganga will assume that your grid certificate is in a certain location
and in a certain format in order to use it. Your grid certificate
therefore needs to be moved and prepared accordingly - which you can do
by following the instructions below.

%-----------------------------------------------------------------------------
\subsubsection{Moving your Grid certificate to your UI}
\label{moving-your-grid-certificate-to-your-ui}
%-----------------------------------------------------------------------------
The first thing to do is move your Grid certificate (the one you got
after following the instructions in Section~\ref{sec:gridcert})
%\href{../getting-on-the-grid/grid-certificate.md}{here}
to the \texttt{\textasciitilde{}/.globus/} directory in your home folder.

\begin{Shaded}
\begin{Highlighting}[]
\NormalTok{$ }\KeywordTok{cd} \NormalTok{~}
\NormalTok{$ }\KeywordTok{pwd}
\NormalTok{[}\KeywordTok{Your} \NormalTok{home directory.]}
\NormalTok{$ }\KeywordTok{mkdir} \NormalTok{.globus}
\NormalTok{$ }\KeywordTok{cp} \NormalTok{[The location of your certificate file.]/[Your certificate filename].p12 ./.globus/.}
\end{Highlighting}
\end{Shaded}

\begin{warningbox}{Certificates on CernVMs}
\emph{If you are using a CernVM and have moved your personal Grid certificate
file to it, you should change the password of the gridpp account so that
no-one else can use it. This can be done in the standard UNIX way with
the passwd command.}
\end{warningbox}

%-----------------------------------------------------------------------------
\subsubsection{Converting your Grid certificate}
\label{converting-your-grid-certificate}
%-----------------------------------------------------------------------------
In order to use your Grid certificate, you need to convert them into
separate certificate and key files. Don't worry, this straightforward
enough to do with the following commands:

\begin{Shaded}
\begin{Highlighting}[]
\NormalTok{$ }\KeywordTok{cd} \NormalTok{~/.globus}
\NormalTok{$ }\KeywordTok{openssl} \NormalTok{pkcs12 -in [Your certificate filename.].p12 -clcerts -nokeys -out usercert.pem}
\KeywordTok{Enter} \NormalTok{Import Password:}
\KeywordTok{MAC} \NormalTok{verified OK}
\NormalTok{$ }\KeywordTok{openssl} \NormalTok{pkcs12 -in [Your certificate filename.].p12 -nocerts -out userkey.pem}
\KeywordTok{Enter} \NormalTok{Import Password:}
\KeywordTok{MAC} \NormalTok{verified OK}
\KeywordTok{Enter} \NormalTok{PEM pass phrase:}
\KeywordTok{Verifying} \NormalTok{- Enter PEM pass phrase:}
\end{Highlighting}
\end{Shaded}

You will then need to change the file permission settings on the two
newly-generated files:

\begin{Shaded}
\begin{Highlighting}[]
\NormalTok{$ }\KeywordTok{chmod} \NormalTok{400 userkey.pem}
\NormalTok{$ }\KeywordTok{chmod} \NormalTok{600 usercert.pem}
\end{Highlighting}
\end{Shaded}

And that's it! You're now ready to use GridPP DIRAC with Ganga. It may
have seemed like a lot of work, but hopefully the next section
%\href{../example-workflow-grid/example-workflow-grid.md}{next section}
will demonstrate it will all have been worth it.


%=============================================================================
\subsection{Checklist}
\label{getting-on-the-grid---checklist}
%=============================================================================

\subsubsection{Your Grid certificate}\label{your-grid-certificate}

\begin{itemize}
\tightlist
\item
  I have requested a grid certificate from the
  \href{http://ngs.ac.uk/ukca}{UK Certificate Authority} (UKCA).
\item
  I know where my nearest
  \href{https://portal.ca.grid-support.ac.uk/caportal/pub/viewralist}{Registration
  Authority} (RA) is.
\item
  I have visited my nearest RA and confirmed my identity.
\item
  I have downloaded my grid certificate .p12 file and installed it in my
  browser.
\item
  I have backed up my grid certificate .p12 file in a secure location.
\end{itemize}

%-----------------------------------------------------------------------------
\subsubsection{Joining a Virtual Organisation -- Checklist}
\label{joining-a-virtual-organisation---checklist}
%-----------------------------------------------------------------------------

\begin{itemize}
\tightlist
\item
  I have submitted a request to join a Virtual Organisation (VO).
\item
  My request has been approved by a VO manager and I have received email
  confirmation of the approval.
\item
  I have followed all of the instructions in the confirmation email.
\end{itemize}

%-----------------------------------------------------------------------------
\subsubsection{First steps with GridPP DIRAC}
\label{first-steps-with-gridpp-dirac}
%-----------------------------------------------------------------------------

\begin{itemize}
\tightlist
\item
  I have accessed the \href{https://dirac.gridpp.ac.uk}{GridPP DIRAC web
  portal} with my Grid certificate-enabled browser.
\item
  I am recognised by the \href{https://dirac.gridpp.ac.uk}{GridPP DIRAC
  web portal} as a \emph{Visitor}.
\item
  I have joined the
  \href{https://mailman.ic.ac.uk/mailman/listinfo/gridpp-dirac-users}{GridPP
  DIRAC mailing list}.
\end{itemize}


%=============================================================================
\subsection{Testing}
\label{getting-on-the-grid---testing}
%=============================================================================

\subsubsection{Your Grid certificate}
\label{your-grid-certificate-testing}

\begin{itemize}
\tightlist
\item
  \textbf{Viewing your certificate details}: visit
  \href{https://portal.ca.grid-support.ac.uk/caportal/cert_owner}{this
  website} with the broswer in which your certificate is installed. You
  should see your grid certificate details displayed.
\end{itemize}

\subsubsection{Joining a Virtual Organisation}
\label{joining-a-virtual-organisation-testing}

\begin{itemize}
\tightlist
\item
  \textbf{Joining the GridPP VO}: Click
  \href{https://voms.gridpp.ac.uk:8443/voms/gridpp/user/search.action}{here}.
  If your request has been approved and confirmed, you should be listed
  as a VO member.
\item
  \textbf{Joining a Regional VO}: Click on
  \href{https://voms.gridpp.ac.uk:8443/voms/vo.londongrid.ac.uk/user/search.action}{\texttt{vo.londongrid.ac.uk}},
  \href{https://voms.gridpp.ac.uk:8443/voms/vo.northgrid.ac.uk/user/search.action}{\texttt{vo.northgrid.ac.uk}},
  \href{https://voms.gridpp.ac.uk:8443/voms/vo.scotgrid.ac.uk/user/search.action}{\texttt{vo.scotgrid.ac.uk}},
  or
  \href{https://voms.gridpp.ac.uk:8443/voms/vo.southgrid.ac.uk/user/search.action}{\texttt{vo.southgrid.ac.uk}}
  (depending on which VO you have attempted to join). If your request
  has been approved and confirmed, you should be listed as a VO member.
\end{itemize}

\subsubsection{First steps with DIRAC}
\label{first-steps-with-dirac-testing}

\begin{itemize}
\tightlist
\item
  \textbf{Accessing the GridPP DIRAC server}: access
  https://dirac.gridpp.ac.uk with your browser. If your grid certificate
  has been successfully installed in your browser, you should be asked
  to identify yourself with the certificate in question. You will then
  see the GridPP DIRAC server homepage. Check the bottom right-hand
  corner - if you can only see \emph{Visitor} and \textbf{not} your
  username and DN, something has gone wrong and you are not (or rather,
  your certificate is not) not registered with the GridPP DIRAC server.
\item
  \textbf{Joining the GridPP DIRAC mailing list}: once you have
  \emph{subscribed} and have \emph{been approved}, you should be able to
  view the
  \href{https://mailman.ic.ac.uk/mailman/roster/gridpp-dirac-users}{subscribers
  list} from the
  \href{https://mailman.ic.ac.uk/mailman/listinfo/gridpp-dirac-users}{list
  homepage} and confirm that you are indeed on it.
\end{itemize}

