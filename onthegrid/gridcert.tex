%=============================================================================
\subsection{Your Grid certificate}
\label{sec:gridcert}
%=============================================================================
Your grid certificate is your passport to the grid. It will give you
access to the vast array of computational resources that GridPP (and the
wLCG) has to offer. As such, getting a grid certificate is an
understandably non-trivial, multi-step process. For example, you will
need to identify yourself to your local Registration Authority (RA) so
that the grid knows who you are.

In concrete terms, a grid certificate is a .p12 file (i.e.~a pkcs12 web
browser certificate file) that you will later convert into a \emph{user
key} file and a \emph{user certificate} file. Encoded according to the
X.509 standard, these are used by the grid to confirm who are you are
and so give you access to grid resources.

%-----------------------------------------------------------------------------
\subsubsection{Requesting a Grid certificate}
\label{requesting-a-grid-certificate}
%-----------------------------------------------------------------------------
Grid certificates in the UK are managed by the
UK e-Science Certificate Authority\footnote{%
See \href{http://ngs.ac.uk/ukca}{http://ngs.ac.uk/ukca}}.
To start the process, you need to choose a web browser that you will have
consistent access to. We recommend Firefox as this process has been
tested and confirmed to work with Firefox on most Operating Systems.
This is because you need to use the same system for both requesting your
certificate and retrieving it when it is ready.

\begin{warningbox}{Temporary logins}
\emph{Do not use a temporary login anywhere when requesting a grid
certificate.}
\end{warningbox}

Using your browser of choice visit the CA portal\footnote{%
See \href{https://portal.ca.grid-support.ac.uk/caportal/}{https://portal.ca.grid-support.ac.uk/caportal/}}
and select the \emph{Request New User Certificate} option. This almost goes
without saying, but make sure you supply a \textbf{valid email address}
which you can access. You will also be asked to do things like supply a
PIN and passwords that you will need later on, so \textbf{make sure you
write everything down}!

\begin{infobox}{Registration Authroties}
\emph{You will need to select a Registration Authority (RA) as part of this
process. If your institution does not have its own RA, select the
nearest on the drop-down menu. You will need to visit the RA in person
with some photographic identification, so don't pick one that is too far
away! If no contact information is listed for a given RA, they will
almost certainly be retrievable using a Search Engine of Your Choice or
via their department's webpage. They will be delighted to hear from you!}
\end{infobox}

Further instructions will then be emailed to you at the email address
you have supplied during the registration process. Once that has
happened you should get a further email from someone at the RA asking
you to visit them in person to complete the validation process.

\begin{warningbox}{Who are you?}
\emph{You may also be asked to supply a letter of recommendation (or, rather,
an email from a suitable authority) explaining why you need to use the
grid and with whom you will be working. If you are unsure about who to
ask for this, please contact us\footnote{%
See \href{https://www.gridpp.ac.uk/contact}{https://www.gridpp.ac.uk/contact}}
and we should be able to help you out.}
\end{warningbox}

%-----------------------------------------------------------------------------
\subsubsection{Installing your Grid certificate in your web browser}
\label{installing-your-grid-certificate-in-your-web-browser}
%-----------------------------------------------------------------------------
Assuming all has gone to plan, you should receive a confirmation email
with a link that will let you download your grid certificate file and
install it in your browser. You will now be able to export and backup
your grid certificate using your browser's certificate management
functionality. This process will vary from browser to browser and from
OS to OS, so consult the
\href{http://www.ngs.ac.uk/ukca/certificates}{UK CA documentation} if in
doubt.

Congratulations - now you can be identified on the grid, you're ready to
join a \term{Virtual Organisation}.
