%=============================================================================
\subsection{Logging on with GridPP DIRAC}
\label{sec:logondirac}
%=============================================================================
There are many ways of accessing and using Grid resources. Larger
organisations - such as the four LHC experiments - have developed their
own frameworks, architectures and mechanisms to enable their members to
run jobs and access experimental data.

One such framework -- \term{DIRAC}~\cite{DIRAC2010} -- is used by
LHCb~\cite{LHCb2008}, but also many other Grid
projects, to manage grid jobs and storage.
You can read more about DIRAC
(Distributed Infrastructure with Remote Agent Control)
on their website\footnote{%
See \href{http://diracgrid.org}{http://diracgrid.org}}
or in~\cite{DIRAC2010},
but for our purposes all you need to
know for now is that DIRAC provides a way for you to access grid
resources without worrying too much about what's going on behind the
scenes.

%-----------------------------------------------------------------------------
\subsubsection{The GridPP DIRAC instance}
\label{the-gridpp-dirac-instance}
%-----------------------------------------------------------------------------
The Imperial College London GridPP Resource Centre (RC) hosts an
instance of DIRAC on behalf of GridPP~\cite{GRIDPPDIRAC2015a,GRIDPPDIRAC2015b}.
The GridPP DIRAC instance is is
capable of serving multiple VOs, providing grid job and data management
capabilities for smaller, non-LHC user communities wishing to make use
of GridPP resources. As a new user, you are automatically registered
with the GridPP DIRAC instance and the Virtual Organisations you have
joined. You can then test out various bits of grid functionality to
determine if grid computing will meet your needs and the needs of your
users.

You can interact with the Grid via the GridPP DIRAC web portal at:

\url{https://dirac.gridpp.ac.uk}

When you access the portal, you should
see yourself listed as a \textbf{Visitor} in drop-down menu in the
bottom-right corner of the browser. Once you have joined one or more
DIRAC-supported Virtual Organisations, you will be able to select which
VO you use DIRAC as using this drop-down menu.

\begin{infobox}{The GridPP DIRAC mailing list}
\emph{You should also join the GridPP DIRAC mailing list to keep informed of
the latest developments and receive notices of any downtime. You can
join the mailing list here.}
\end{infobox}

So you've accessed the GridPP DIRAC web portal. Congratulations!
However, as discussed, we'll be using GridPP DIRAC via the Ganga
interface. In order to do that, you'll need to install your Grid
certificate on your local machine - and the instructions for
doing this are in the next section.
