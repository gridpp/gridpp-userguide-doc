%=============================================================================
\subsection{Getting help}
\label{sec:help}
%=============================================================================
There are many ways of getting help and support if you run into problems
while working through the \emph{GridPP UserGuide}. If you don't happen
to have a GridPP expert in the office down the corridor, you can try the
methods described below.

%-----------------------------------------------------------------------------
\subsubsection{Check the troubleshooting guide}
\label{check-the-troubleshooting-guide}
%-----------------------------------------------------------------------------

We've added a
\hyperref[sec:troubleshooting]{short troubleshooting guide}
for problems that users have come across that we
know are specific to particular systems, generally raised via the
\href{http://github.com/gridpp/user-guides/issues}{GitHub Issues page}.
It might be worth checking here first for anything obvious.

%-----------------------------------------------------------------------------
\subsubsection{Googling the error}
\label{googling-the-error}
%-----------------------------------------------------------------------------
We can't possibly account for every error a user might encounter when
working through the \emph{UserGuide}, so on encountering a problem your
first port of call should be sticking the error message into your Search
Engine of Choice.

\begin{infobox}{Errors on the Internet}
\emph{This is actually a pretty good approach to software development in
general. Thanks to vibrant, enthusastic communities like those at
StackExchange many common computing gotchas have been documented and
solved on the World Wide Web - so it's always worth checking!}
\end{infobox}

%-----------------------------------------------------------------------------
\subsubsection{Issue tracking via GitHub}
\label{issue-tracking-via-github}
%-----------------------------------------------------------------------------
The easiest way to report problems, make suggestions, or submit comments
about the \emph{UserGuide} is by raising an \textbf{issue} on the
\emph{GridPP UserGuide}
\href{http://github.com/GridPP/user-guides}{GitHub repository}. Simply
log in to GitHub, visit the \emph{UserGuide}
\href{https://github.com/gridpp/user-guides/issues}{issues page} and
click on the \href{https://github.com/gridpp/user-guides/issues/new}{New
issue} button.

\begin{infobox}{Submitting an issue to GitHub}
\emph{Provide as much information as you can when raising an issue. You can
also use the MarkDown format to create hyperlinks and add formatting to
your issue.}
\end{infobox}

We'll then have a public record of the issue which we can then aim to
solve as soon as we can. It's also possible to link issues to the
\href{https://help.github.com/articles/using-pull-requests/}{pull
requests} that fix them.

\begin{infobox}{Watching GitHub repositories}
\emph{Don't forget to Watch the repository too. You can do this by going to
the repository, signing in with your GitHub account, and clicking on the
Watch button at the top-right of the page. You'll then be kept
up-to-date with issues and new versions as the UserGuide evolves over
time.}
\end{infobox}

%-----------------------------------------------------------------------------
\subsubsection{Mailing lists}
\label{mailing-lists}
%-----------------------------------------------------------------------------
A great way to tap into the expertise represented by the GridPP
Collaboration is to join one of the mailing lists in the table below.
You'll need a valid email address, but if you've read the
\hyperref[sec:prerequisites]{prerequisites} you know that already.

%______________________________________________________________________________
\begin{table}[htbp]
\caption{\label{tab:mailinglists}The GridPP support mailing lists.}
\lineup
\begin{tabular}{@{}llc}
\br
\centre{1}{$\quad$List        $\quad$} & 
\centre{1}{$\quad$Description $\quad$} &
\centre{1}{$\quad$Subscribe   $\quad$} \\
\mr
\texttt{GRIDPP-USERS} &
A list for announcements and discussions & 
\href{https://www.jiscmail.ac.uk/cgi-bin/webadmin?SUBED1=GRIDPP-USERS\&A=1}{JISCMail} \\
 &
aimed at UK Grid users. &
\\
\texttt{GRIDPP-SUPPORT} &
A list for discussion and support & 
\href{https://www.jiscmail.ac.uk/cgi-bin/webadmin?SUBED1=GRIDPP-SUPPORT\&A=1}{JISCMail} \\
 &
aimed at UK Grid users. &
\\
\br
\end{tabular}
\end{table}
%______________________________________________________________________________

\begin{warningbox}{Using the mailing lists}
\emph{These mailing lists are public, so keep it nice people!}
\end{warningbox}

%-----------------------------------------------------------------------------
\subsubsection{Contact us}
\label{contact-us}
%-----------------------------------------------------------------------------
Finally, if none of the other methods yield results, drop us a line
using the details here:

\href{https:///www.gridpp.ac.uk/contact/}{https:///www.gridpp.ac.uk/contact/}
