%=============================================================================
\subsection{Prerequisites}
\label{sec:prerequisites}
%=============================================================================
While we want to make the grid accessible to everyone, there are some
things you are going to need - and are going to need to know - in order
to take full advantage of the resources on offer. Let's go through these
now.

\begin{itemize}
\tightlist
\item
  \textbf{A valid email address}: This sounds kind of obvious - and who
  \emph{doesn't} have an email address these days? - but you'll need a
  valid email address from which you can send and receive emails.
\end{itemize}

\begin{hintbox}{Which email address should I use?}
\emph{If you can, use your institutional or organisational email account (such
as that given to you by your school, university, or company) as this
will make life a little easier when it comes to granting you access to
grid resources.}
\end{hintbox}

\begin{itemize}
\tightlist
\item
  \textbf{A GitHub account}: GridPP is an Open Source project. Much of
  the software used by the GridPP Collaboration is hosted on our
  \href{http://github.com/gridpp}{GitHub repository} - including the
  \emph{UserGuide} itself. It means we can track developments and issues
  in a public forum and so maximise collaboration opportunities. You can
  sign up for a free GitHub account on \href{http://github.com}{their
  website}.
\end{itemize}

\begin{infobox}{Online code repositories}
\emph{GitHub isn't the only online code repository available. For example,
BitBucket also offers a similar git-based versioning system. CERN, too,
have their own git system. GitHub is used because they allow unlimited
public repositories with an unlimited number of collaborators - which is
what GridPP is all about. BitBucket, on the other hand, offer an
unlimited number of private repositories for users, but limit the number
who can collaborate. Please get in touch if you'd like to know more.}
\end{infobox}

\begin{itemize}
\item
  \textbf{Experience with the command line}: The command line allows you
  to type instructions into your computer in order to get it to do
  things for you, rather than relying on clicking on icons, buttons, and
  other graphical elements of a software package. In his guide,
  \href{http://www.learnenough.com/command-line-tutorial}{Learn Enough
  Command Line to be Dangerous}, Michael Hartl uses a nice analogy with
  magic. While it is \emph{technically} possible to use the Grid without
  using the command line (using, for example, a web browser to access
  specific Grid systems), using the command line is infinitely easier
  and gives you much, much more flexibility. Hartl's
  \href{http://www.learnenough.com/command-line-tutorial}{tutorial}, is
  well worth following if you've not used it before (or even if you
  have!).
\item
  \textbf{A text editor}: we'll be writing scripts - series of commands
  to be executed one after the other - and for this you'll need a text
  editor of some description. Emacs, Vim, Vi - whatever you feel most
  comfortable with. Vim, for example, allows you to edit text from the
  command line.
\item
  \textbf{Programming with Python}: Once we start getting fancy with the
  Grid, we're going to use the Python programming language (via an
  Application Programming Interface) to do a lot of the work for us. As
  such, some familiarity with Python will be handy. There are plenty of
  (free!) online tutorials available that can get you started. We'll
  provide plenty of examples too, so don't panic!
\item
  \textbf{Contact with GridPP}: Before we can let you loose on GridPP's
  vast computing resources, it'd be nice to know who you are and what
  you're doing with them. In fact, this is a requirement of the UK Grid
  policy. You may already be in contact with a GridPP representative at
  your local institution - if not, feel free to
  \href{https://www.gridpp.ac.uk/contact/}{drop us a line}.
\item
  \textbf{A Scientific Linux 6 command line with CVMFS access}: This
  will either be provided by your friendly GridPP contact (see above) or
  via a \href{../gridpp-cernvm/gridpp-cernvm.html}{GridPP CernVM}, a
  Virtual Machine made by \href{http://cern.home}{CERN} that you can run
  yourself. The \href{https://cernvm.cern.ch/}{CernVM-File System},
  a.k.a. CernVM-FS or CVMFS, gives you (and any grid node, for that
  matter) instant access to all sorts of software \emph{without having
  to install anything}. So it's worth sorting out! You can find
  instructions for creating a
  \href{../gridpp-cernvm/gridpp-cernvm.html}{GridPP CernVM} in
  \href{../gridpp-cernvm/gridpp-cernvm.html}{this appendix}.
\end{itemize}

Got all of that? Good. Now let's look at the
conventions used by the \emph{GridPP UserGuide}.
