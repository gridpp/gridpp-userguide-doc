%%%%%%%%%%%%%%%%%%%%%%%%%%%%%%%%%%%%%%%%%%%%%%%%%%%%%%%%%%%%%%%%%%%%%%%%%%%%%%%
\section{Using Grid Data in Your Workflow}
\label{sec:griddatarunning}
%%%%%%%%%%%%%%%%%%%%%%%%%%%%%%%%%%%%%%%%%%%%%%%%%%%%%%%%%%%%%%%%%%%%%%%%%%%%%%%
The example workflow we've used so far isn't particularly sophisticated,
but it does allow us to demonstrate the final key concept we'll look at
here: incorporating Grid-based data in your workflows. Below we'll go
through:

\begin{itemize}
\tightlist
\item
  Putting the initial input data onto the Grid;
\item
  Using that data as the input to your workflow;
\item
  Writing the output data from your workflow to the Grid;
\item
  Retrieving the output from the Grid.
\end{itemize}

You'll need to have worked through Section~\ref{sec:gridrunning}
first -- mainly because we actually only need to tweak a few lines to
achieve what we want!

%=============================================================================
\subsection{Putting our input data onto the Grid}
\label{putting-our-input-data-onto-the-grid}
%=============================================================================
The first thing to do is put the ZIP file containing the raw frame data
into your user area on the DFC. You know how to do this (if not, have
another look at Section~\ref{sec:griddata})
so now we're getting into the realm of user areas, VOs, etc.
we're not going to give the explicit commands for this part. We'll also
leave you to come up with a LFN for the file, and choose which SE to use
(you might have a favourite by now).

\begin{itemize}
\tightlist
\item
  Download the ZIP file to your working directory;
\item
  Upload it to your user area in your VO's area in the DFC;
\item
  \textbf{Note down the LFN you assigned the file}. You'll need this
  below.
\end{itemize}

\begin{infobox}{Structuring your LFNs}
\emph{While the directories don't strictly exist with LFNs, it's useful to
keep things organised with sensible structuring/naming conventions. Use
the DFC CLI to create directories in your user area as required.}
\end{infobox}

%=============================================================================
\subsection{Getting data from the Grid}
\label{getting-data-from-the-grid}
%=============================================================================
Thanks to Ganga, there's actually not much to using a Grid-hosted data
file as input. All you need to do is add a \texttt{DiracFile} to the
job's \texttt{inputputfile} list with the LFN as the input argument. So
Ada's modified \texttt{dirac\_job.py} script would have the line:

\begin{Shaded}
\begin{Highlighting}[]
\KeywordTok{j.inputfiles} \NormalTok{= [ DiracFile(}\StringTok{'LFN:/.../CERNatschool_backgroundrad_dataset.zip'}\NormalTok{) ]}
\end{Highlighting}
\end{Shaded}

Replacing the \code{...} in the LFN with the full path.
The job will now retrieve the ZIP file from whichever Storage Element 1)
has a replica of the file and/or 2) is closest to the site running the
job to the working directory, just as it would with the
\texttt{LocalFile}.

\begin{infobox}{Replica loaction and management}
\emph{In fact, GridPP DIRAC will work out where is best to send your job based
upon where you have replicas of the file (i.e.~which SEs you
added/replicated it on). So once you're into optimisation territory,
replica management is something to think about.}
\end{infobox}

%=============================================================================
\subsection{Writing data to the Grid}
\label{writing-data-to-the-grid}
%=============================================================================
What about the output data? If you have an intermediary data layer
(i.e.~output that is used as input for another job/workflow) you may
wish to write the output to the Grid. This is possible with a few
tweaks, but there's a slight subtlety: GridPP DIRAC will assign LFNs for
your job output based on the DIRAC job ID and an LFN base specified in
your \texttt{.gangarc} file. This can be set with something like the
following:

\begin{Shaded}
\begin{Highlighting}[]
\NormalTok{[}\KeywordTok{DIRAC}\NormalTok{]}
\KeywordTok{DiracLFNBase} \NormalTok{= /gridpp/user/a/ada.lovelace}
\end{Highlighting}
\end{Shaded}

\begin{warningbox}{Preparing to submit}
\emph{Make sure you set this before starting Ganga and submitting your job(s).}
\end{warningbox}

Specifying which files get written to the Grid is then pretty similar to
specifying the input files - switch the \texttt{LocalFile} to
\texttt{DiracFile}:

\begin{Shaded}
\begin{Highlighting}[]
\KeywordTok{j.outputfiles} \NormalTok{= [ DiracFile(}\StringTok{'output_images.tar'}\NormalTok{) ]}
\end{Highlighting}
\end{Shaded}

With these changes made (and maybe a change of job name), you can now
submit your job.

%=============================================================================
\subsection{Retrieving the output from the Grid}
\label{retrieving-the-output-from-the-grid}
%=============================================================================
You already know how to
retrieve files from the Grid. The only extra detail you'll need to know
is the DIRAC job ID. \textbf{This is different to the job ID in Ganga}.
Both can be obtained with the following commands within Ganga:

\begin{Shaded}
\begin{Highlighting}[]
\KeywordTok{Ganga} \NormalTok{In [X]: j.id}
\KeywordTok{Ganga} \NormalTok{Out [X]: 1}

\KeywordTok{Ganga} \NormalTok{In [X]: j.backend.id}
\KeywordTok{Ganga} \NormalTok{Out [X]: 1234567}
\end{Highlighting}
\end{Shaded}

(i.e.~the DIRAC ID will have many more digits.)

The DIRAC ID will determine the LFN the output files are assigned. So
once the job has finished running, you should end up with something like
this:

\begin{Shaded}
\begin{Highlighting}[]
\NormalTok{$ }\KeywordTok{dirac-dms-filecatalog-cli} 
\KeywordTok{Starting} \NormalTok{FileCatalog client}

\KeywordTok{File} \NormalTok{Catalog Client }\OtherTok{$Revision}\NormalTok{: 1.17 }\OtherTok{$Date}\NormalTok{: }
            
\KeywordTok{FC}\NormalTok{:/}\KeywordTok{>} \NormalTok{cd gridpp/user/a/ada.lovelace/}
\KeywordTok{FC}\NormalTok{:/gridpp/user/a/ada.lovelace}\KeywordTok{>}\NormalTok{ls}
\KeywordTok{1234}
\KeywordTok{FC}\NormalTok{:/}\KeywordTok{>} \NormalTok{ls 1234}
\KeywordTok{1234567}
\KeywordTok{FC}\NormalTok{:/}\KeywordTok{>} \NormalTok{ls 1234/1234567}
\KeywordTok{frames.json}
\KeywordTok{output_images.tar}
\end{Highlighting}
\end{Shaded}

So the full LFN for the image archive is:

\begin{Shaded}
\begin{Highlighting}[]
\KeywordTok{LFN}\NormalTok{:/gridpp/user/a/ada.lovelace/1234/1234567/output_images.tar}
\end{Highlighting}
\end{Shaded}

This can be used to retrieve the file in the ways we have described
already - or used as an \texttt{inputfile} to another job.

So there we go - we've completely Grid-ified our example workflow. You
should now have all of the tools you need to start making your own
workflows Grid-ready. Of course, there's a lot more that can be done and
we'll mention some of the more advanced topics in the
next section.
But you should have plenty to get your teeth into for now!

%=============================================================================
\subsection{Checklist}
\label{using-grid-based-data-in-your-workflow---checklist}
%=============================================================================

\begin{itemize}
\tightlist
\item
  I can successfully modify the Grid workflow example to:
\item
  use an input \texttt{DiracFile} to with an lfn corresponding to data I
  have uploaded to the Grid;
\item
  write the output data to my user area on the DFC.
\item
  I can set the output LFN base in my Ganga configuration file;
\item
  I can run the modified Grid workflow example and retrieve the output
  from the Grid.
\end{itemize}


%=============================================================================
\subsection{Testing}
\label{using-grid-based-data-in-your-workflow---testing}
%=============================================================================

\begin{itemize}
\tightlist
\item
  \textbf{Successful running of the CERN@school example job}: Does
  Figure~\ref{fig:localrunningframe} look familiar? It really should do by now.
\end{itemize}

