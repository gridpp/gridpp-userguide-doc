%%%%%%%%%%%%%%%%%%%%%%%%%%%%%%%%%%%%%%%%%%%%%%%%%%%%%%%%%%%%%%%%%%%%%%%%%%%%%%%
\section{First Steps: Hello World(s)!}
\label{sec:helloworlds}
%%%%%%%%%%%%%%%%%%%%%%%%%%%%%%%%%%%%%%%%%%%%%%%%%%%%%%%%%%%%%%%%%%%%%%%%%%%%%%%
It's something of a tradition to start any new computing activity with
an exercise that produces the phrase,
``\href{https://en.wikipedia.org/wiki/\%22Hello,_World!\%22_program}{Hello,
World!}'' with whatever you're doing. And it's not a tradition we'll be
breaking with now. Distributed computing, however, is all about doing
things on a bigger scale - so we'll be saying ``Hello'' to many worlds
all at the same time. We'll do this using the
\href{http://ganga.readthedocs.io}{Ganga} toolsuite. By the end of this
chapter you'll have used Ganga to submit multiple jobs with a single
command and check their output.

\begin{infobox}{Jobs}
\emph{A job is the term we use to describe a task, or set of tasks, we run on
our local machine, our cluster's machines, or the grid's Worker Nodes
(WNs). We'll come back to these concepts later in the UserGuide.}
\end{infobox}

We won't be using the grid yet - they will run on your local machine -
but thanks to Ganga and the other tools used by GridPP making the switch
to grid running is pretty straightforward.

Ready? Let's say ``Hello!''

\subsection{Starting Ganga}\label{starting-ganga}

\href{http://ganga.readthedocs.io}{Ganga} is a Python-based toolkit used
for submitting jobs and managing data on the grid. You can read more
about it on its \href{https://ganga.web.cern.ch/ganga/}{CERN page here}.
The code is all on \href{https://github.com/ganga-devs/ganga}{GitHub},
of course. Crucially, Ganga is available via CVMFS so \emph{you don't
even have to install it}. On your terminal with CVMFS access, Ganga can
be started by simply typing

\begin{Shaded}
\begin{Highlighting}[]
\NormalTok{$ }\KeywordTok{source} \NormalTok{/cvmfs/ganga.cern.ch/runGanga.sh}
\end{Highlighting}
\end{Shaded}

After various welcome messages have been presented, you should see the
Ganga command prompt:

\begin{Shaded}
\begin{Highlighting}[]
\KeywordTok{Ganga} \NormalTok{In [1]:}
\end{Highlighting}
\end{Shaded}

\begin{infobox}{iPython}
If you're familiar with iPython, this prompt style should look very
familiar!
\end{infobox}

\begin{hintbox}{Numbered prompts}
\emph{The number you see in the Ganga (iPython) prompt is the number of the
command that you've executed in the terminal. This is great for
interactive running, but rubbish for writing user guides. We'll replace
this is with an} \code{X} \emph{in what follows.}
\end{hintbox}

\begin{hintbox}{Quitting Ganga}
\emph{To quit Ganga, press}
\texttt{Ctrl-d}
\emph{and then type}
\texttt{y}
\emph{or press Enter.}
\end{hintbox}

You will be using Ganga a lot. You may want to create an alias for the
Ganga start command in your \textasciitilde{}/.bashrc file.

All good so far? Great. Now you're ready to submit your first
\textbf{job}.

\subsection{Submitting a Hello, World!
job}\label{submitting-a-hello-world-job}

Ganga has an iPython-esque command line interface for real-time job
management. Things like jobs are modelled using Python objects. So
creating and submitting a job is as simple as this:

\begin{Shaded}
\begin{Highlighting}[]
\KeywordTok{Ganga} \NormalTok{In [X]: j = Job()}
\KeywordTok{Ganga} \NormalTok{In [X]: j.submit() }
\end{Highlighting}
\end{Shaded}

You should see an output that looks like something like this:

\begin{Shaded}
\begin{Highlighting}[]
\KeywordTok{INFO}     \NormalTok{submitting job X}
\KeywordTok{INFO}     \NormalTok{job X status changed to }\StringTok{"submitting"}
\KeywordTok{INFO}     \NormalTok{Preparing Executable application.}
\KeywordTok{INFO}     \NormalTok{Created shared directory: [temporary directory name]}
\KeywordTok{INFO}     \NormalTok{Preparing subjobs}
\KeywordTok{INFO}     \NormalTok{submitting job X to Local backend}
\KeywordTok{INFO}     \NormalTok{job X status changed to }\StringTok{"submitted"}
\end{Highlighting}
\end{Shaded}

What's going on here? Well, the \texttt{Job()} object has a bunch of
default settings that, on instantiation, create a Hello, World! job that
submits to the ``Local'' backend - i.e.~the machine you are running on.

\begin{infobox}{Back-ends}
The back-end is wherever you want your jobs to run - your local machine,
your local computing cluster, or the grid (via GridPP DIRAC). The beauty
of Ganga is that you have the same interface for whichever you are using
- which makes switching between them a case of tweaking some
configuration files.
\end{infobox}

In the time it has taken to read the above, you should see the following
output (press return if not).

\begin{Shaded}
\begin{Highlighting}[]
\KeywordTok{INFO}     \NormalTok{job X status changed to }\StringTok{"running"}
\KeywordTok{INFO}     \NormalTok{Job X Running PostProcessor hook}
\KeywordTok{INFO}     \NormalTok{job X status changed to }\StringTok{"completed"}
\KeywordTok{INFO}     \NormalTok{removing: [temporary directory name]}
\end{Highlighting}
\end{Shaded}

Your job has finished. You can check this - and the status of any other
jobs - using the \texttt{jobs} command, which produces a neat little
summary table of all of your jobs:

\begin{Shaded}
\begin{Highlighting}[]
\KeywordTok{Ganga} \NormalTok{In [X]: jobs}
\KeywordTok{Ganga} \NormalTok{Out [X]: }
\KeywordTok{Registry} \NormalTok{Slice: jobs (1 object)}
\end{Highlighting}
\end{Shaded}

OK, so your job has run and finished, but where is the famous phrase
that signifies success? Well, Ganga manages your job output for you in a
series of output directories. More on this later, but you can peek at
the output with the following command from Ganga:

\begin{Shaded}
\begin{Highlighting}[]
\KeywordTok{Ganga} \NormalTok{In [X]: j.peek(}\StringTok{'stdout'}\NormalTok{, }\StringTok{'more'}\NormalTok{)}
\KeywordTok{Hello} \NormalTok{World}
\end{Highlighting}
\end{Shaded}

\begin{infobox}{Reading the output}
\emph{We've used the} \code{more} \emph{program to read the output, but you can specify your
own if you like\ldots{}}
\end{infobox}

Ta da! You've submitted, run, completed, and checked the output from,
your first grid-like job. OK, so it didn't run on the grid this time,
but thanks to Ganga the process isn't actually that different.

\begin{infobox}{Local running for testing workflows}
\emph{In fact the ability to run grid-like jobs locally is very useful for
testing your workflow out before unleashing it on the grid\ldots{}}
\end{infobox}

Finally, to tidy up:

\begin{Shaded}
\begin{Highlighting}[]
\KeywordTok{Ganga} \NormalTok{In [X]: j.remove()}
\KeywordTok{INFO}     \NormalTok{removing job X}
\end{Highlighting}
\end{Shaded}

You can check with the \texttt{jobs} command that the job has gone from
the list.

Congratulations - you've submitted your first job! But we can do better
than that: with distributed computing, the idea is to break your problem
into bits and tackle them with multiple jobs: \emph{divide and conquer}.
Let's see how easy this is to do with Ganga.

\subsection{Submitting the Hello, World(s)!
jobs}\label{submitting-the-hello-worlds-jobs}

We're going to use a Python script to generate and submit multiple jobs
to the Local backend with Ganga. First, use your favourite editor to
create the following script (which we will call
\texttt{hello\_worlds.py}):

\begin{Shaded}
\begin{Highlighting}[]
\NormalTok{$ }\KeywordTok{cat} \NormalTok{hello_worlds.py}
\KeywordTok{worlds} \NormalTok{= [}\StringTok{'Mercury'}\NormalTok{, }\StringTok{'Venus'}\NormalTok{, }\StringTok{'Mars'}\NormalTok{, }\StringTok{'Earth'}\NormalTok{, }\StringTok{'Jupiter'}\NormalTok{, }\StringTok{'Saturn'}\NormalTok{, \\}
\StringTok{'Uranus'}\NormalTok{, }\StringTok{'Neptune'}\NormalTok{, }\StringTok{'Pluto'}\NormalTok{]}

\KeywordTok{for} \KeywordTok{world} \NormalTok{in worlds:}
    \KeywordTok{j} \NormalTok{= Job()}
    \KeywordTok{j.name} \NormalTok{= }\StringTok{"hello_%s"} \NormalTok{% (world.lower())}
    \KeywordTok{j.application.args} \NormalTok{= [}\StringTok{"Hello, %s!"} \NormalTok{% (world)]}
    \KeywordTok{j.submit}\NormalTok{()}
\end{Highlighting}
\end{Shaded}

\begin{hintbox}{Two terminals}
\emph{You may want to have two terminals running in your working directory --
one to run Ganga in, and one to write scripts in. This will save having
to quit Ganga each time you want to edit a script with a command-line
editor (e.g.~vim).}
\end{hintbox}

There are a few things to note here:

\begin{itemize}
\tightlist
\item
  We have given each job a \texttt{name} using the script. This will
  make life easier later on once the jobs have finished.
\item
  The \texttt{executable} used is still the default (\texttt{echo}), but
  now we have supplied varying arguments for the different jobs.
\end{itemize}

Ganga can then run this script with the \texttt{execfile} command. The
script uses a \texttt{for} loop to create the jobs and submit them:

\begin{Shaded}
\begin{Highlighting}[]
\KeywordTok{Ganga} \NormalTok{In [X]: execfile(}\StringTok{'hello_worlds.py'}\NormalTok{)}
\NormalTok{[}\KeywordTok{...} \NormalTok{updates on the job submission ...]}
\end{Highlighting}
\end{Shaded}

All being well, all nine jobs will run and complete. Using \texttt{jobs}
to find the job ID, you can look at the output as before:

\begin{Shaded}
\begin{Highlighting}[]
\KeywordTok{Ganga} \NormalTok{In [X]: jobs(7)}\KeywordTok{.peek}\NormalTok{(}\StringTok{'stdout'}\NormalTok{, }\StringTok{'more'}\NormalTok{)}
\KeywordTok{Hello}\NormalTok{, Neptune!}
\end{Highlighting}
\end{Shaded}

\subsubsection{Job manipulation tips and
tricks}\label{job-manipulation-tips-and-tricks}

Of course, now you're able to create and submit potentially huge numbers
of jobs, you may want to think about how to keep on top of which jobs
are which, how to remove jobs, etc. This is where Ganga really comes in
to its own. For example:

\begin{itemize}
\tightlist
\item
  \textbf{Selecting jobs by name}: remember how we gave each job a name?
  Well, this allows us to select the jobs we want in one go:
\end{itemize}

\begin{Shaded}
\begin{Highlighting}[]
\KeywordTok{Ganga} \NormalTok{In [X]: my_jobs = jobs.select(name=}\StringTok{'hello_*'}\NormalTok{)}

\KeywordTok{Ganga} \NormalTok{In [X]: my_jobs}
\KeywordTok{Ganga} \NormalTok{Out [6]: }
\KeywordTok{Registry} \NormalTok{Slice: jobs.select(minid=}\StringTok{'None'}\NormalTok{, maxid=}\StringTok{'None'}\NormalTok{, name=}\StringTok{"None"}\NormalTok{) }\KeywordTok{(9} \NormalTok{objects}\KeywordTok{)}
\end{Highlighting}
\end{Shaded}

You can now use the \texttt{my\_jobs} object to do things to your jobs,
such as \texttt{submit}, \texttt{copy}, \texttt{resubmit}, etc.

\begin{hintbox}{Tab complete}
\emph{Use tab complete to see what's possible with the}
\code{my\_jobs} \emph{you've selected.}
\end{hintbox}

\begin{itemize}
\tightlist
\item
  \textbf{Removing multiple jobs}: To tidy up all the jobs in one go,
  use the slice you've created with the \texttt{select} command:
\end{itemize}

\begin{Shaded}
\begin{Highlighting}[]
\KeywordTok{Ganga} \NormalTok{In [X]: my_jobs.remove()}
\KeywordTok{INFO}     \NormalTok{removing job X}
\NormalTok{[}\KeywordTok{...}\NormalTok{]}
\KeywordTok{INFO}     \NormalTok{removing job X+8}
\end{Highlighting}
\end{Shaded}

You can verify this has been successful with the \texttt{jobs} command.

So there we go - your first multiple job submission with Ganga.
Obviously we are going to need to incorporate more complicated features
to adapt your workflow for grid running - using your own executables and
software libraries, uploading input data, extracting the output, etc. -
but hopefully you can see how we might go about this using Ganga and,
ultimately, the Grid.

Now take a look at the following checklist to make sure
you've got everything from this chapter nailed. Then we'll look at a
\hyperref[sec:localrunning]{more complicated workflow} in
Section~\ref{sec:localrunning}.

%=============================================================================
\subsection{Checklist}
\label{first-steps-hello-worlds---checklist}
%=============================================================================

\begin{itemize}
\tightlist
\item
  I can start Ganga from my command line;
\item
  I can submit a simple ``Hello, World!'' job using the Ganga default
  \texttt{Job()};
\item
  I can give a job a \texttt{name};
\item
  I can write a script that creates and submits multiple jobs;
\item
  I can select a group of jobs based on the name I assigned them on
  creation;
\item
  I can remove multiple jobs with a single command operating on my
  selection of jobs.
\end{itemize}


%=============================================================================
\subsection{Testing}
\label{first-steps-hello-worlds---testing}
%=============================================================================

\begin{itemize}
\item
  \textbf{Running Ganga from the command line}: if you have successfully
  run Ganga, you should now have the following in your \texttt{\$HOME}
  directory:

\begin{Shaded}
\begin{Highlighting}[]
\NormalTok{$ }\KeywordTok{cd} \OtherTok{$HOME}
\NormalTok{$ }\KeywordTok{ls} \NormalTok{~/.gangarc}
\KeywordTok{/home/alovelace/.gangarc}
\NormalTok{$ }\KeywordTok{ls} \NormalTok{~/gangadir}
\KeywordTok{repository}  \NormalTok{shared  thread_trace.html  workspace}
\end{Highlighting}
\end{Shaded}
\item
  \textbf{Looking at the output from local Ganga jobs}: assuming you
  haven't removed them (you can always re-run them again if you have),
  you should be able to find the actual output files from your jobs in
  your \texttt{gangadir}. So for user \texttt{alovelace}'s job 0, the
  output can be found in:

\begin{Shaded}
\begin{Highlighting}[]
\NormalTok{$ }\KeywordTok{ls} \OtherTok{$HOME}\NormalTok{/gangadir/workspace/alovelace/LocalXML/0/output/}
\KeywordTok{__jobstatus__} \NormalTok{stdout stderr __syslog__}
\end{Highlighting}
\end{Shaded}

  You should see something similar (and you can look at the output in
  \texttt{stdout} directly if you like!).
\end{itemize}

