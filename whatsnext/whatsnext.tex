%%%%%%%%%%%%%%%%%%%%%%%%%%%%%%%%%%%%%%%%%%%%%%%%%%%%%%%%%%%%%%%%%%%%%%%%%%%%%%%
\section{What's Next?}
\label{sec:whatsnext}
%%%%%%%%%%%%%%%%%%%%%%%%%%%%%%%%%%%%%%%%%%%%%%%%%%%%%%%%%%%%%%%%%%%%%%%%%%%%%%%

%=============================================================================
\subsection{Uploading your software to CVMFS}
\label{uploading-your-software-to-cvmfs}
%=============================================================================
To start with, and if your software package/packages is/are small
enough, you can probably get away with uploading your software as
\texttt{LocalFile}s with your Grid job (perhaps using an archive file).
\emph{In fact, you may prefer this approach as http-based CVMFS repositories
are world-readable.}

However, at some point you and/or your Virtual Organisation are going to
want your own CVMFS repository. For small, UK-based VOs the best way to
do this is on the
\href{https://www.gridpp.ac.uk/wiki/RAL_Tier1_CVMFS}{RAL Tier-1
Stratum-0}. \href{mailto:info@gridpp.ac.uk}{Contact us} to find out more
about doing this - access to the repository is governed by your Grid
certificate.

In short, the process of uploading your software amounts to:

\begin{itemize}
\tightlist
\item \textbf{Generating a proxy with DIRAC}: as this will be used to
determine who you are and which repository you are accessing;
\item \textbf{Logging in to the repository}: You can then log in to your CVMFS
repository with:

\begin{Shaded}
\begin{Highlighting}[]
\NormalTok{$ }\KeywordTok{gsissh} \NormalTok{-p 1975 cvmfs-upload01.gridpp.rl.ac.uk}
\KeywordTok{Last} \NormalTok{login: [Date/time] from [hostname]}

   \KeywordTok{_.} \NormalTok{_.   o}\KeywordTok{|} \KeywordTok{_} \NormalTok{._}
  \KeywordTok{(_|(_||_|||(_)|} \KeywordTok{|}
       \KeywordTok{|}

          \KeywordTok{Location}\NormalTok{: r89.harwell.europe hpd r89rack137}
            \KeywordTok{Branch}\NormalTok{: cc34/cvmfs-uploader (sandbox}\KeywordTok{)}
         \KeywordTok{Archetype}\NormalTok{: ral-tier1}
       \KeywordTok{Personality}\NormalTok{: cvmfs-uploader}
  \KeywordTok{Operating} \NormalTok{System: sl640-x86_64}
     \KeywordTok{Snapshot} \NormalTok{Date: 2016-10-26}

\NormalTok{[}\DataTypeTok{\{vo-name\}}\KeywordTok{sgm@cvmfsXXXXX} \NormalTok{~]$ cd cvmfs_repo}
\KeywordTok{bin} \NormalTok{lib code README.md}
\end{Highlighting}
\end{Shaded}

\item
  \textbf{Retrieve your software}: You can now place your software in
  the repository, arranging it however works best for you. If your code
  is hosted in an online Git-based repository, simply
  \texttt{git\ clone} it straight to an appropriate directory.
\end{itemize}

\begin{infobox}{Looking at other CVMFS repositories}
\emph{You can take inspiration from other VOs. For example, you can browse the
ATLAS experiment's (CERN-hosted) CVFMS repository with:}
code{\$ ls /cvmfs/atlas.cern.ch/} \emph{and so on}.
\end{infobox}

\begin{warningbox}{Upload times}
Once put into the repository, it can be several hours before it becomes
available on the worker nodes - it's not instant. Make sure your
software is well-tested before putting it up - CVMFS is not appropriate
for software under development!
\end{warningbox}

However, once it \emph{is} on there, it's available everywhere. Which is
nice.

%=============================================================================
\subsection{Advanced DIRAC functionality}
\label{advanced-dirac-functionality}
%=============================================================================
DIRAC has a great deal of functionality of its own, particularly when
you start looking at the Python API. However, Ganga provides a nice
wrapper for much of this so you shouldn't need to touch it. You can find
out more on the \href{http://diracgrid.org/}{DIRAC homepage}, and check
out the source code on their \href{https://github.com/DIRACGrid}{GitHub
page}.

%=============================================================================
\subsection{Advanced Ganga functionality}
\label{advanced-ganga-functionality}
%=============================================================================
Likewise, there is a lot more to Ganga than we have covered here. Ganga
has its own documentation page:

\url{http://ganga.readthedocs.io}

which features things like:

\begin{itemize}
\tightlist
\item Configuration;
\item Job manipulation;
\item Splitters;
\item Post-processors;
\item Queues;
\item Etc.
\end{itemize}

The \href{https://github.com/ganga-devs/ganga}{GitHub repository} is
also worth watching for the latest updates and developments, as well as
raising any bugs or problems you may have in the
\href{https://github.com/ganga-devs/ganga/issues}{Issues} section.

\subsection{And finally\ldots{}}\label{and-finally}

Moving your workflow to the Grid won't necessarily be straightforward,
but at GridPP we're here to help -- if you've got a problem, just ask!
To keep up to date with the
UserGuide, watch the
\href{https://github.com/gridpp/user-guides}{\emph{UserGuide} GitHub
repository} to be notified of
\href{https://github.com/gridpp/user-guides/issues}{Issues} and Pull
Requests relating to additions or improvements to the \emph{UserGuide}.

Many thanks for reading so far, and happy Gridding!
