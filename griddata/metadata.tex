%=============================================================================
\subsection{First steps with the DIRAC metadata}
\label{first-steps-with-the-dirac-metadata-functionality}
%=============================================================================

%-----------------------------------------------------------------------------
\subsubsection{Finding files using metadata}
\label{finding-files-using-metadata}
%-----------------------------------------------------------------------------
When you're uploading vast amounts of data, it's nice to be able to find
it later. \textbf{Metadata} - data \emph{about the data} - can help with
this. DIRAC allows you to assign metadata such as strings, integers, and
floating point numbers to files and directories (via their Logical File
Names in the DIRAC File Catalog). You can then query the DFC to return a
list of the files you want.

For example, once you have sourced your DIRAC environment, generated a
proxy, and started the DFC CLI, you can find all files associated with
the \texttt{UserGuide} experiment like so:

\begin{Shaded}
\begin{Highlighting}[]
\KeywordTok{FC}\NormalTok{:/}\KeywordTok{>} \NormalTok{find / experiment=UserGuide}
\KeywordTok{Query}\NormalTok{: \{}\StringTok{'experiment'}\NormalTok{: }\StringTok{'UserGuide'}\NormalTok{\}}
\KeywordTok{/gridpp/userguide/WELCOME.md}
\KeywordTok{QueryTime} \NormalTok{0.98 sec}
\end{Highlighting}
\end{Shaded}

We have assigned the value \texttt{UserGuide} to the file
\texttt{WELCOME.md} for the \texttt{experiment} element or index. The
\texttt{find} command in the DFC CLI performs the query for us.

\begin{Shaded}
\begin{Highlighting}[]
\KeywordTok{FC}\NormalTok{:/}\KeywordTok{>} \NormalTok{help find}
 \KeywordTok{Find} \NormalTok{all files satisfying the given metadata information }
    
        \KeywordTok{usage}\NormalTok{: find [-q] [-D] }\KeywordTok{<}\NormalTok{path}\KeywordTok{>} \KeywordTok{<}\NormalTok{meta_name}\KeywordTok{>}\NormalTok{=}\KeywordTok{<}\NormalTok{meta_value}\KeywordTok{>} \NormalTok{[}\KeywordTok{<}\NormalTok{meta_name}\KeywordTok{>}\NormalTok{=}\KeywordTok{<}\NormalTok{meta_value}\KeywordTok{>}\NormalTok{]}
    
\KeywordTok{FC}\NormalTok{:/}\KeywordTok{>} \NormalTok{exit}
\end{Highlighting}
\end{Shaded}

In our query above, \texttt{\textless{}path\textgreater{}} was
\texttt{/} (i.e.~search the entire catalog from the base directory),
\texttt{\textless{}meta\_name\textgreater{}} was \texttt{experiment}
(i.e.~a metadata string index indicating to which experiment the data
belongs), and \texttt{\textless{}meta\_value\textgreater{}} was
\texttt{UserGuide} (OK, so the \texttt{UserGuide} isn't really an
experiment - at least not in the scientific sense - but you get the
idea!).

\begin{infobox}{Getting help with the DFC CLI}
\emph{You can get a list of all of the available commands in the DFC CLI by
using the help command. To list the instructions for a given command (as
above), type} \texttt{help {[}command{]}}.
\end{infobox}

There is only one file belonging to the \texttt{UserGuide} experiment in
the DFC, and it's a pretty harmless MarkDown file. But you can hopefully
see how, particularly when we start using multiple metadata indices with
different \emph{types}, DIRAC's metadata functionality is going to be
pretty useful.

%-----------------------------------------------------------------------------
\subsubsection{Assigning metadata to a file}
\label{assigning-metadata-to-a-file}
%-----------------------------------------------------------------------------
We can also use the DFC CLI to \emph{assign} metadata to our files.
Let's create a file with our favourite text editor and upload it to the
grid using the DFC CLI:

\begin{Shaded}
\begin{Highlighting}[]
\NormalTok{$ }\KeywordTok{vim} \NormalTok{TODO.md}
\NormalTok{$ }\KeywordTok{cat} \NormalTok{TODO.md}
\KeywordTok{ToDo}
\NormalTok{====}
\KeywordTok{*} \NormalTok{Email Charles re. engine}
\KeywordTok{*} \NormalTok{Re-do punchcards}
\KeywordTok{*} \NormalTok{Write to Dad}
\NormalTok{$ }\KeywordTok{dirac-dms-filecatalog-cli} 
\KeywordTok{Starting} \NormalTok{FileCatalog client}

\KeywordTok{File} \NormalTok{Catalog Client }\OtherTok{$Revision}\NormalTok{: 1.17 }\OtherTok{$Date}\NormalTok{: }
            
\KeywordTok{FC}\NormalTok{:/}\KeywordTok{>} \NormalTok{add /gridpp/user/a/ada.lovelace/TODO.md TODO.md UKI-LT2-QMUL2-disk}
\KeywordTok{File} \NormalTok{/gridpp/user/a/ada.lovelace/TODO.md successfully uploaded...}
\end{Highlighting}
\end{Shaded}

We can now set the \texttt{owner} index for the LFN using the
\texttt{meta\ set} command:

\begin{Shaded}
\begin{Highlighting}[]
\KeywordTok{FC}\NormalTok{:/}\KeywordTok{>} \NormalTok{meta set /gridpp/user/a/ada.lovelace/TODO.md owner ada.lovelace}
\KeywordTok{/gridpp/user/a/ada.lovelace/TODO.md} \NormalTok{owner ada.lovelace}
\end{Highlighting}
\end{Shaded}

We can now find the file again using the \texttt{find} command:

\begin{Shaded}
\begin{Highlighting}[]
\KeywordTok{FC}\NormalTok{:/}\KeywordTok{>} \NormalTok{find / owner=ada.lovelace}
\KeywordTok{Query}\NormalTok{: \{}\StringTok{'owner'}\NormalTok{: }\StringTok{'ada.lovelace'}\NormalTok{\}}
\KeywordTok{/gridpp/user/a/ada.lovelace/TODO.md}
\KeywordTok{QueryTime} \NormalTok{0.01 sec}
\end{Highlighting}
\end{Shaded}

As we've said before, the DFC CLI is useful for small-scale operations
on your data. Hopefully, though, you can start to appreciate the power
of \textbf{metadata} when it comes to organising your data and
performing analyses on it.

The most important thing for the moment, though, is that we can now put
data on the Grid (i.e.~on a Storage Element). This means we can use it
in our Grid jobs without needing to upload with our job as an
\texttt{inputfile}. We'll now complete making our example workflow fully
Grid-enabled in the next section.
