%%%%%%%%%%%%%%%%%%%%%%%%%%%%%%%%%%%%%%%%%%%%%%%%%%%%%%%%%%%%%%%%%%%%%%%%%%%%%%%
\section{Putting Data on the Grid}
\label{sec:griddata}
%%%%%%%%%%%%%%%%%%%%%%%%%%%%%%%%%%%%%%%%%%%%%%%%%%%%%%%%%%%%%%%%%%%%%%%%%%%%%%%
We've now moved the local example workflow to the Grid.
However,
we've still only used data that's been present on our local system, and
we've manually retrieved the output to our local system. To harness the
full power of the Grid, we'll need to put data on it. We'll use tools
provided with DIRAC to do this, namely:

\begin{itemize}
\tightlist
\item The DIRAC File Catalog Command Line Interface (DFC CLI);
\item The DIRAC command line tools;
\item Some first steps with the DFC's metadata functionality.
\end{itemize}

First, though, let's look at some basic concepts in grid-based data
management.

%=============================================================================
\subsection{Storage Elements, File Catalogs, and Replicas}
\label{storage-elements-file-catalogs-and-replicas}
%=============================================================================
The first thing to wrap one's head around with distributed computing is
the notion that you don't really need to care about where your data is
stored. You may well be used to this concept if you've dealt with
cloud-based storage services such as Dropbox, Google Drive, or even
Amazon S3 storage. Your files are on one or more servers
\emph{somewhere}, and all that you need to know are the file names and
the directories that they're in to access them later.

It's the same with the grid. You upload your files to a grid
\textbf{Storage Element} (SE) and label them with a \textbf{Logical File
Name} (LFN) that gets registered in something called a \textbf{File
Catalog}. If you make copies of a particular file - a \textbf{replica} -
on one or more additional SEs, the locations of these replicas are
recorded in the File Catalog too.

\begin{infobox}{Storage Elements and replicas}
\emph{With most cloud-based storage services, you won't even really care about
the Storage Elements (or their non-grid equivalents, whatever the they
happen to be called) and file replicas. However, when considering
running grid jobs at a particular grid site, the location of your
replicas can matter (you'll want to make sure your data is available at
sites that will run jobs for your Virtual Organisation). We'll come back
to all of these concepts - and provide concrete examples - later.}
\end{infobox}

The GridPP DIRAC system provides a suite of tools to help you manage all
of this. If you're familiar with UNIX-based file systems you should find
it all pretty straightforward. We'll start with the
DIRAC File Catalog Command Line Interface.

%=============================================================================
\subsection{The DFC Command Line Interface}
\label{the-dfc-command-line-interface}
%=============================================================================
The DIRAC File Catalog (DFC) Command Line Interface (CLI), a.k.a. the
\textbf{DFC CLI}, provides a way of interacting with DIRAC's File
Catalog via - you guessed it - the command line. The DFC CLI lets you
manually upload and download files to Storage Elements (SEs), browse the
DFC associated with your Virtual Organisation (VO), create and remove
directories in the DFC, and manage the replicas associated with each
entry in the DFC.

\begin{infobox}{Using the DFC CLI}
\emph{The DFC CLI is great for small-scale tasks such as creating and tweaking
test data sets, but ultimately we will want to use scripts to help
coordinate large-scale upload operations and managing metadata
(i.e.~data about the data).}
\end{infobox}

%-----------------------------------------------------------------------------
\subsubsection{Getting started with the DFC CLI}
\label{getting-started-with-the-dfc-cli}
%-----------------------------------------------------------------------------

\term{Accessing the DFC CLI}

The DFC CLI is accessed via a DIRAC command, so we'll need to source our
DIRAC environment and generate a DIRAC proxy.

\begin{Shaded}
\begin{Highlighting}[]
\NormalTok{$ }\KeywordTok{source} \NormalTok{/cvmfs/ganga.cern.ch/dirac_ui/bashrc}
\NormalTok{$ }\KeywordTok{dirac-proxy-init} \NormalTok{-g gridpp_user -M}
\KeywordTok{Generating} \NormalTok{proxy... }
\KeywordTok{Enter} \NormalTok{Certificate password: }\CommentTok{# Enter your grid certificate password...}
\KeywordTok{.}
\KeywordTok{.} \NormalTok{[}\KeywordTok{Proxy} \NormalTok{information-based output.]}
\KeywordTok{.}
\end{Highlighting}
\end{Shaded}

\begin{warningbox}{Which VO?}
\emph{If you wish to use a different VO, replace gridpp with the name of the
VO in the commands in this section.}
\end{warningbox}

The DFC CLI is then started with the following DIRAC command:

\begin{Shaded}
\begin{Highlighting}[]
\NormalTok{$ }\KeywordTok{dirac-dms-filecatalog-cli} 
\KeywordTok{Starting} \NormalTok{FileCatalog client}

\KeywordTok{File} \NormalTok{Catalog Client }\OtherTok{$Revision}\NormalTok{: 1.17 }\OtherTok{$Date}\NormalTok{: }
            
\KeywordTok{FC}\NormalTok{:/}\KeywordTok{>}
\end{Highlighting}
\end{Shaded}

\begin{infobox}{DIRAC command groupings}
\emph{We'll come back to the DIRAC command line tools in the next section, but
the} \code{dirac-dms-} \emph{at the start of the command refers to the DIRAC Data
Management System tools. All DIRAC commands are grouped in this way
which, combined with tab completion, can be very handy for finding the
command you're looking for!}
\end{infobox}

The \texttt{FC:/\textgreater{}} at the command prompt tells you that
you're in the DFC CLI. You can now explore the DFC using commands that
are very similar to those used with a typical UNIX file system. Let's do
this now.

%-----------------------------------------------------------------------------
\subsubsection{Finding your user space in the DFC}
\label{finding-your-user-space-in-the-dfc}
%-----------------------------------------------------------------------------
Let's start by listing the root directories in the DFC, which will give
us a list of the Virtual Organisations supported by GridPP DIRAC:

\begin{Shaded}
\begin{Highlighting}[]
\KeywordTok{FC}\NormalTok{:/}\KeywordTok{>} \NormalTok{ls}
\KeywordTok{cernatschool.org}
\KeywordTok{gridpp}
\KeywordTok{vo.londongrid.ac.uk}
\KeywordTok{vo.northgrid.ac.uk}
\KeywordTok{vo.scotgrid.ac.uk}
\KeywordTok{vo.southgrid.ac.uk}
\end{Highlighting}
\end{Shaded}

We're using GridPP DIRAC as a member of \texttt{gridpp} VO, so let's
move into that directory.

\begin{Shaded}
\begin{Highlighting}[]
\KeywordTok{FC}\NormalTok{:/}\KeywordTok{>} \NormalTok{cd gridpp/user}
\end{Highlighting}
\end{Shaded}

If one hasn't been created for you already, you can create your own user
space on the VO's File Catalog like so:

\begin{Shaded}
\begin{Highlighting}[]
\KeywordTok{FC}\NormalTok{:/gridpp/user}\KeywordTok{>} \NormalTok{cd a}
\KeywordTok{FC}\NormalTok{:/gridpp/user/a}\KeywordTok{>} \NormalTok{mkdir ada.lovelace}
\KeywordTok{FC}\NormalTok{:/gridpp/user/a}\KeywordTok{>} \NormalTok{chmod 755 ada.lovelace}
\KeywordTok{FC}\NormalTok{:/gridpp/user/a}\KeywordTok{>} \NormalTok{ls -la}
\KeywordTok{drwxr-xr-x} \NormalTok{0 ada.lovelace gridpp_user 0 2015-12-16 10:24:54 ada.lovelace }
\KeywordTok{FC}\NormalTok{:/gridpp/user/a}\KeywordTok{>} \NormalTok{exit}
\end{Highlighting}
\end{Shaded}

\begin{infobox}{Your DIRAC username}
\emph{If you don't know your DIRAC username (which should be used as your user
directory), exit the DFC CLI and use the dirac-proxy-info command.}
\end{infobox}

\begin{infobox}{Listing files}
\emph{Using the} \code{-la} \emph{option with the}
\code{ls} \emph{command works just as it does with the
normal command line, allowing you to see file owners, groups (VOs),
permissions, etc.}
\end{infobox}

\begin{warningbox}{File permissions}
\emph{Don't forget to change the file permissions on your files so that other
users can't modify them.}
\end{warningbox}

You've now got your own space on the GridPP DFC. Let's put some files in
it.

%-----------------------------------------------------------------------------
\subsubsection{Uploading files}
\label{uploading-files}
%-----------------------------------------------------------------------------
Firstly, we'll need a file to upload. Any file will do, but to keep
things simple let's create one in a temporary directory:

\begin{Shaded}
\begin{Highlighting}[]
\NormalTok{$ }\KeywordTok{cd} \NormalTok{~}
\NormalTok{$ }\KeywordTok{mkdir} \NormalTok{tmp}\KeywordTok{;} \KeywordTok{cd} \NormalTok{tmp}
\NormalTok{$ }\KeywordTok{vim} \NormalTok{TEST.md }\CommentTok{# Or whichever editor you use...}
\NormalTok{$ }\KeywordTok{cat} \NormalTok{TEST.md}
\CommentTok{#Hello Grid!}
\KeywordTok{This} \NormalTok{is a test **MarkDown file**.}
\end{Highlighting}
\end{Shaded}

Next we'll need to know which \textbf{Storage Elements} are available to
us.

\begin{infobox}{Storage Elements}
\emph{Storage Elements} ``are physical sites where data are stored and
accessed, for example, physical file systems, disk caches or
hierarchical mass storage systems. Storage Elements manage storage and
enforce authorization policies on who is allowed to create, delete and
access physical files. They enforce local as well as Virtual
Organization policies for the use of storage resources. They guarantee
that physical names for data objects are valid and unique on the storage
device(s), and they provide data access. A storage element is an
interface for grid jobs and grid users to access underlying storage
through the Storage Resource Management protocol (SRM), the Globus Grid
FTP protocol, and possibly other interfaces as well.''

\emph{Credit: Open Science Grid (2012)}
\end{infobox}

We can list the available SEs with the following DIRAC command:

\begin{Shaded}
\begin{Highlighting}[]
\NormalTok{$ }\KeywordTok{dirac-dms-show-se-status} 
\KeywordTok{SE}                           \NormalTok{ReadAccess WriteAccess RemoveAccess CheckAccess }
\NormalTok{=============================================================================}
\NormalTok{[}\KeywordTok{...} \NormalTok{more disks ...]}
\KeywordTok{UKI-LT2-QMUL2-disk}           \NormalTok{Active     Active      Unknown      Unknown     }
\NormalTok{[}\KeywordTok{...} \NormalTok{more disks ...]}
\KeywordTok{UKI-NORTHGRID-LIV-HEP-disk}   \NormalTok{Active     Active      Unknown      Unknown}
\NormalTok{[}\KeywordTok{...} \NormalTok{more disks ...]}
\end{Highlighting}
\end{Shaded}

While we don't need to know the details of where and how our data will
be stored on an SE, we do need to know its name. We'll use the
\texttt{UKI-LT2-QMUL2-disk} SE for now. We add the file to the DFC as
follows using the \texttt{add} command, which takes the following
arguments:

\begin{verbatim}
add <LFN> <Local file name> <SE name>
\end{verbatim}

where:

\begin{itemize}
\tightlist
\item
  \texttt{\textless{}LFN\textgreater{}} is the \textbf{Logical File
  Name} (LFN) of the file in the DFC. This can either be relative to
  your current position in the DFC (which can be found with the
  \texttt{pwd} command in the DFC CLI), or made absolute by preceeding
  the name with a slash \texttt{/};
\item
  \texttt{\textless{}Local\ file\ name\textgreater{}} should be the name
  of the local file you want to upload. Again, this can be relative to
  wherever you were on your local system when you started the DFC CLI,
  or the absolute path to the file on your local system;
\item
  \texttt{\textless{}SE\ name\textgreater{}} is the name of the SE as
  retrived from the \texttt{dirac-dms-show-se-status} command.
\end{itemize}

Let's add our file to the grid now.

\begin{Shaded}
\begin{Highlighting}[]
\NormalTok{$ }\KeywordTok{dirac-dms-filecatalog-cli}
\KeywordTok{Starting} \NormalTok{FileCatalog client}

\KeywordTok{File} \NormalTok{Catalog Client }\OtherTok{$Revision}\NormalTok{: 1.17 }\OtherTok{$Date}\NormalTok{: }
            
\KeywordTok{FC}\NormalTok{:/}\KeywordTok{>} \NormalTok{cd /gridpp/user/a/ada.lovelace}
\KeywordTok{FC}\NormalTok{:/gridpp/user/a/ada.lovelace}\KeywordTok{>} \NormalTok{mkdir tmp}
\KeywordTok{FC}\NormalTok{:/gridpp/user/a/ada.lovelace}\KeywordTok{>} \NormalTok{cd tmp}
\KeywordTok{FC}\NormalTok{:/gridpp/user/a/ada.lovelace}\KeywordTok{>} \NormalTok{add TEST.md TEST.md UKI-LT2-QMUL2-disk}

\KeywordTok{File} \NormalTok{/gridpp/user/a/ada.lovelace/tmp/TEST.md successfully uploaded...}
\KeywordTok{FC}\NormalTok{:/gridpp/user/a/ada.lovelace/tmp}\KeywordTok{>}\NormalTok{ls -la}
\KeywordTok{-rwxrwxr-x} \NormalTok{1 ada.lovelace gridpp_user 47 2015-12-16 11:47:28 TEST.md}
\end{Highlighting}
\end{Shaded}

And there we go! Your first file has been uploaded to a Storage Element
on the grid. Have a biscuit. You've earned it.

%-----------------------------------------------------------------------------
\subsubsection{Replicating files}
\label{replicating-files}
%-----------------------------------------------------------------------------
Part of the joy of using the grid is being able to distribute
computational tasks to different sites. However, if you want to look at
the same data with a different task at different sites in an efficient
manner, ideally you'd need copies of that data at those sites. This
strategy also makes sense from a backup/redundancy perspective. We can
achieve this on the grid by using \emph{replicas}.

\begin{infobox}{Replicas}
\emph{A replica is a copy of a given file that is located on a different
Storage Element (SE). The file is identified by its Logical File Name
(LFN) in the DIRAC File Catalog (DFC). Associated with each LFN entry is
a list of SEs where replicas of the file can be found.}
\end{infobox}

To list the locations of replicas for a given file catalog entry, we use
the \texttt{replicas} command in the DFC CLI:

\begin{verbatim}
replicas <LFN>
\end{verbatim}

so continuing with our example:

\begin{Shaded}
\begin{Highlighting}[]
\KeywordTok{FC}\NormalTok{:/gridpp/user/a/ada.lovelace/tmp}\KeywordTok{>}\NormalTok{replicas TEST.md}
\KeywordTok{lfn}\NormalTok{: /gridpp/user/a/ada.lovelace/tmp/TEST.md}
\end{Highlighting}
\end{Shaded}

We replicate files with the \texttt{replicate} command:

\begin{verbatim}
replicate <LFN> <SE name>
\end{verbatim}

Let's replicate our test file to the Liverpool disk and check that the
replica list has been updated:

\begin{Shaded}
\begin{Highlighting}[]
\KeywordTok{FC}\NormalTok{:/gridpp/user/a/ada.lovelace/tmp}\KeywordTok{>}\NormalTok{replicate TEST.md UKI-NORTHGRID-LIV-HEP-disk}
\end{Highlighting}
\end{Shaded}

Replicas can be removed with the \texttt{rmreplica} command:

\begin{verbatim}
rmreplica <LFN> <SE name>
\end{verbatim}

Let's remove the Liverpool disk replica:

\begin{Shaded}
\begin{Highlighting}[]
\KeywordTok{FC}\NormalTok{:/gridpp/user/a/ada.lovelace/tmp}\KeywordTok{>}\NormalTok{rmreplica TEST.md UKI-NORTHGRID-LIV-HEP-disk}
\KeywordTok{lfn}\NormalTok{: /gridpp/user/a/ada.lovelace/tmp/TEST.md}
\KeywordTok{Replica} \NormalTok{at UKI-NORTHGRID-LIV-HEP-disk moved to Trash Bin}
\end{Highlighting}
\end{Shaded}

Finally, we can remove a file completely using the (somewhat familiar)
\texttt{rm} command:

\begin{verbatim}
rm <LFN>
\end{verbatim}

Let's tidy up our test file:

\begin{Shaded}
\begin{Highlighting}[]
\KeywordTok{FC}\NormalTok{:/gridpp/user/a/ada.lovelace/tmp}\KeywordTok{>}\NormalTok{rm TEST.md}
\KeywordTok{lfn}\NormalTok{: /gridpp/user/a/ada.lovelace/tmp/TEST.md}
\KeywordTok{File} \NormalTok{/gridpp/user/a/ada.lovelace/tmp/TEST.md removed from the catalog}
\end{Highlighting}
\end{Shaded}

%-----------------------------------------------------------------------------
\subsubsection{Downloading files}
\label{downloading-files}
%-----------------------------------------------------------------------------
Finally, we can download files using the DFC CLI with the \texttt{get}
command:

\begin{verbatim}
get <LFN> [<local directory>]
\end{verbatim}

Note that the local directory argument is optional. Let's download a
test file from the \texttt{gridpp} examples directory now:

\begin{Shaded}
\begin{Highlighting}[]
\KeywordTok{FC}\NormalTok{:/}\KeywordTok{>} \NormalTok{get /gridpp/userguide/WELCOME.md ./.}
\KeywordTok{FC}\NormalTok{:/}\KeywordTok{>} \NormalTok{exit}
\NormalTok{$ }\KeywordTok{cat} \NormalTok{WELCOME.md}
\CommentTok{#Welcome to GridPP!}

\KeywordTok{It} \NormalTok{looks like your download has worked. Congratulations!}
\NormalTok{$ }\KeywordTok{rm} \NormalTok{WELCOME.md}
\end{Highlighting}
\end{Shaded}

As we said earlier, the DFC CLI is only useful for small-scale
operations. On our way to scaling up, we can look at starting to
automate our workflows using scripts. In the next section we'll look at
how the DIRAC command line tools can help with this.


%=============================================================================
\subsection{The DIRAC Command Line Tools}
\label{the-dirac-command-line-tools}
%=============================================================================
So you've mastered the DFC Command Line Interface. Great stuff. What
you'll have probably noticed is that, while it's great for small-scale
operations, it's not ideal for doing things with lots of files on any
sort of scale. We will therefore want to take a look at the
\textbf{DIRAC command line tools} for data management.

\begin{infobox}{The command line tools}
\emph{All of the DIRAC command line tools start with}
\code{dirac-}. \emph{The data management tools start with}
\code{dirac-dms-}, \emph{as in Data Management System.
Press the tab key after typing} \texttt{dirac-dms-}
\emph{to see all of the available commands.}
\end{infobox}

Why are these commands useful? Well, it means you can use
\emph{scripting} to automate large-scale tasks involving many files.
There are many ways to \emph{script} the DIRAC (or indeed any command
line) commands. You've probably got your own preferred method that
reflects your coding background. For the purposes of the UserGuide,
we'll use simple bits of \textbf{Python} code (along with Python-based
file management libraries) to generate some simple \texttt{bash} scripts
that can then be run to perform the DIRAC operations we want to perform.

\begin{infobox}{Scripting DIRAC commands}
\emph{Of course, bash experts will be able to write scripts that perform all
of the operations below purely in bash. This is left as an exercise for
the reader - answers on a punch card please! (Also, we'll be using
Python for the DIRAC Python API, so it's not a bad thing to use Python
at this stage!)}
\end{infobox}

%-----------------------------------------------------------------------------
\subsubsection{Uploading files}
\label{uploading-files-tools}
%-----------------------------------------------------------------------------
The DIRAC file upload command takes the following form:

\begin{verbatim}
$ dirac-dms-add-file <LFN> <FILE> <SE>
\end{verbatim}

where:

\begin{itemize}
\tightlist
\item \texttt{\textless{}LFN\textgreater{}} is the Logical File Name
(LFN) of the entry for the file in the DIRAC File Catalog (DFC);
\item \texttt{\textless{}FILE\textgreater{}} is the path to the file on your
local machine, and;
\item \texttt{\textless{}SE\textgreater{}} is the name
of the destination Storage Element (SE).
\end{itemize}

\begin{infobox}{Finding SE names}
\emph{Remember, you can find the names of the available SEs with the
dirac-dms-show-se-status command.}
\end{infobox}

Suppose we have a number of files on our local machine in
\texttt{/home/gridpp/mydata/} that we want to upload to the grid. The
following Python code will generate a \texttt{bash} script that will
upload them to one of the Queen Mary Storage Elements:

\begin{Shaded}
\begin{Highlighting}[]
\NormalTok{$ cat make_upload_script.py}
\CommentTok{#!/usr/bin/env python}
\CommentTok{# -*- coding: utf-8 -*-}

\ImportTok{import} \NormalTok{os, glob}

\NormalTok{data_path }\OperatorTok{=} \StringTok{'/home/gridpp/mydata'}

\NormalTok{lfn_dir }\OperatorTok{=} \StringTok{'/gridpp/user/a/ada.lovelace/mydata/'}

\NormalTok{se }\OperatorTok{=} \StringTok{'UKI-LT2-QMUL2-disk'}

\NormalTok{s }\OperatorTok{=} \StringTok{"#!/bin/bash}\CharTok{\textbackslash{}n}\StringTok{"}

\ControlFlowTok{for} \NormalTok{my_file }\OperatorTok{in} \BuiltInTok{sorted}\NormalTok{(glob.glob(data_path }\OperatorTok{+} \StringTok{"/*"}\NormalTok{)):}
    \NormalTok{base_name  }\OperatorTok{=} \NormalTok{os.path.basename(my_file)}
    \NormalTok{upload_lfn }\OperatorTok{=} \NormalTok{os.path.join(lfn_dir, base_name)}
    \NormalTok{s }\OperatorTok{+=} \StringTok{"dirac-dms-add-file }\SpecialCharTok{%s}\StringTok{ }\SpecialCharTok{%s}\StringTok{ }\SpecialCharTok \NormalTok{(upload_lfn, my_file, se)}

\ControlFlowTok{with} \BuiltInTok{open}\NormalTok{(}\StringTok{"upload_script.sh"}\NormalTok{, }\StringTok{"w"}\NormalTok{) }\ImportTok{as} \NormalTok{sf:}
    \NormalTok{sf.write(s)}
\end{Highlighting}
\end{Shaded}

After you've generated a proxy and sourced the DIRAC environment, you
can run the generated script as follows:

\begin{Shaded}
\begin{Highlighting}[]
\NormalTok{$ }\KeywordTok{python} \NormalTok{make_upload_script.py}
\NormalTok{$ }\KeywordTok{chmod} \NormalTok{a+x upload_script.sh}
\NormalTok{$ }\KeywordTok{.} \KeywordTok{upload_script.sh}
\end{Highlighting}
\end{Shaded}

The results of this will, of course, depend on the contents of
\texttt{/home/gridpp/mydata/}, but all being well you should see the
message:

\begin{verbatim}
Successfully uploaded file to UKI-LT2-QMUL2-disk
\end{verbatim}

(or whichever SE you specified in your Python code) after each file has
been uploaded.

\begin{infobox}{Using Screen}
\emph{If you're uploading a lot of files, you may wish to consider using
something like the screen tool so that you can log off your terminal
session and come back to it later.}
\end{infobox}

And there we go! Multiple file uploads, all registered in the DIRAC File
Catalog, using a DIRAC command line tool and a bit of (admitedly
slightly clumsy) coding.

%-----------------------------------------------------------------------------
\subsubsection{Replicating files}
\label{replicating-files-tools}
%-----------------------------------------------------------------------------
Now, as we did with the DFC CLI, we can also make replicas of files,
list information about the replicas of a given file, and remove replicas
with the following command line tools:

\begin{verbatim}
dirac-dms-replicate-lfn <LFN> <SE>
dirac-dms-lfn-replicas <LFN>
dirac-dms-remove-replicas <LFN> <SE>
\end{verbatim}

Likewise, we can take the same approach with\ldots{}

%-----------------------------------------------------------------------------
\subsubsection{Downloading and removing files}
\label{downloading-and-removing-files-tools}
%-----------------------------------------------------------------------------

\begin{verbatim}
dirac-dms-get-file <LFN>
dirac-dms-remove-files <LFN>
\end{verbatim}

i.e.~the DIRAC command line tools exist for these operations. However,
getting information from the DFC about which files you would like to
replicate, download, remove, etc. is non-trivial when taking the command
line approach. This is especially true if you're writing scripts.

One approach is to use the \textbf{metadata} functionality the DIRAC
File Catalog provides to find files of interest.

\begin{infobox}{Metadata}
\term{Metadata} \emph{is data about the data. By assigning metadata to the files we
upload to the DIRAC File Catalog, we can perform queries that will
select only the files we are interested in. It also helps us to manage
our data. We'll find out more about the DFC's metadata functionality
later.}
\end{infobox}

The \texttt{dirac-dms-find-lfns} command finds LFNs based on the DFC
path and metadata query supplied as options. For example, to find all
files in the DFC that have been assigned to the experiment
\texttt{UserGuide}, we can type:

\begin{Shaded}
\begin{Highlighting}[]
\KeywordTok{dirac-dms-find-lfns} \NormalTok{Path=/ }\StringTok{"experiment=UserGuide"}
\NormalTok{\{}\StringTok{'experiment'}\NormalTok{: }\StringTok{'UserGuide'}\NormalTok{\}}
\KeywordTok{/gridpp/userguide/WELCOME.md}
\end{Highlighting}
\end{Shaded}

\texttt{experiment} here is the \textbf{metadata element} or
\textbf{index}. This is a string assigned to the file's LFN that, in
this case, has the value \texttt{UserGuide}. We can use the results of
this to download the files we want.

\begin{Shaded}
\begin{Highlighting}[]
\NormalTok{$ }\KeywordTok{dirac-dms-get-file} \NormalTok{/gridpp/userguide/WELCOME.md}
\NormalTok{\{}\StringTok{'Failed'}\NormalTok{: }\DataTypeTok{\{\}}\NormalTok{,}
 \StringTok{'Successful'}\NormalTok{: \{}\StringTok{'/gridpp/userguide/WELCOME.md'}\NormalTok{: }\StringTok{'/home/gridpp/tmp/WELCOME.md'}\NormalTok{\}\}}
\NormalTok{$ }\KeywordTok{cat} \NormalTok{WELCOME.md}
\CommentTok{#Welcome to GridPP!}

\KeywordTok{It} \NormalTok{looks like your download has worked. Congratulations!}
\end{Highlighting}
\end{Shaded}

Let's take a closer look at the DFC's metadata functionality using the
DFC CLI.


%=============================================================================
\subsection{First steps with the DIRAC metadata}
\label{first-steps-with-the-dirac-metadata-functionality}
%=============================================================================

%-----------------------------------------------------------------------------
\subsubsection{Finding files using metadata}
\label{finding-files-using-metadata}
%-----------------------------------------------------------------------------
When you're uploading vast amounts of data, it's nice to be able to find
it later. \textbf{Metadata} - data \emph{about the data} - can help with
this. DIRAC allows you to assign metadata such as strings, integers, and
floating point numbers to files and directories (via their Logical File
Names in the DIRAC File Catalog). You can then query the DFC to return a
list of the files you want.

For example, once you have sourced your DIRAC environment, generated a
proxy, and started the DFC CLI, you can find all files associated with
the \texttt{UserGuide} experiment like so:

\begin{Shaded}
\begin{Highlighting}[]
\KeywordTok{FC}\NormalTok{:/}\KeywordTok{>} \NormalTok{find / experiment=UserGuide}
\KeywordTok{Query}\NormalTok{: \{}\StringTok{'experiment'}\NormalTok{: }\StringTok{'UserGuide'}\NormalTok{\}}
\KeywordTok{/gridpp/userguide/WELCOME.md}
\KeywordTok{QueryTime} \NormalTok{0.98 sec}
\end{Highlighting}
\end{Shaded}

We have assigned the value \texttt{UserGuide} to the file
\texttt{WELCOME.md} for the \texttt{experiment} element or index. The
\texttt{find} command in the DFC CLI performs the query for us.

\begin{Shaded}
\begin{Highlighting}[]
\KeywordTok{FC}\NormalTok{:/}\KeywordTok{>} \NormalTok{help find}
 \KeywordTok{Find} \NormalTok{all files satisfying the given metadata information }
    
        \KeywordTok{usage}\NormalTok{: find [-q] [-D] }\KeywordTok{<}\NormalTok{path}\KeywordTok{>} \KeywordTok{<}\NormalTok{meta_name}\KeywordTok{>}\NormalTok{=}\KeywordTok{<}\NormalTok{meta_value}\KeywordTok{>} \NormalTok{[}\KeywordTok{<}\NormalTok{meta_name}\KeywordTok{>}\NormalTok{=}\KeywordTok{<}\NormalTok{meta_value}\KeywordTok{>}\NormalTok{]}
    
\KeywordTok{FC}\NormalTok{:/}\KeywordTok{>} \NormalTok{exit}
\end{Highlighting}
\end{Shaded}

In our query above, \texttt{\textless{}path\textgreater{}} was
\texttt{/} (i.e.~search the entire catalog from the base directory),
\texttt{\textless{}meta\_name\textgreater{}} was \texttt{experiment}
(i.e.~a metadata string index indicating to which experiment the data
belongs), and \texttt{\textless{}meta\_value\textgreater{}} was
\texttt{UserGuide} (OK, so the \texttt{UserGuide} isn't really an
experiment - at least not in the scientific sense - but you get the
idea!).

\begin{infobox}{Getting help with the DFC CLI}
\emph{You can get a list of all of the available commands in the DFC CLI by
using the help command. To list the instructions for a given command (as
above), type} \texttt{help {[}command{]}}.
\end{infobox}

There is only one file belonging to the \texttt{UserGuide} experiment in
the DFC, and it's a pretty harmless MarkDown file. But you can hopefully
see how, particularly when we start using multiple metadata indices with
different \emph{types}, DIRAC's metadata functionality is going to be
pretty useful.

%-----------------------------------------------------------------------------
\subsubsection{Assigning metadata to a file}
\label{assigning-metadata-to-a-file}
%-----------------------------------------------------------------------------
We can also use the DFC CLI to \emph{assign} metadata to our files.
Let's create a file with our favourite text editor and upload it to the
grid using the DFC CLI:

\begin{Shaded}
\begin{Highlighting}[]
\NormalTok{$ }\KeywordTok{vim} \NormalTok{TODO.md}
\NormalTok{$ }\KeywordTok{cat} \NormalTok{TODO.md}
\KeywordTok{ToDo}
\NormalTok{====}
\KeywordTok{*} \NormalTok{Email Charles re. engine}
\KeywordTok{*} \NormalTok{Re-do punchcards}
\KeywordTok{*} \NormalTok{Write to Dad}
\NormalTok{$ }\KeywordTok{dirac-dms-filecatalog-cli} 
\KeywordTok{Starting} \NormalTok{FileCatalog client}

\KeywordTok{File} \NormalTok{Catalog Client }\OtherTok{$Revision}\NormalTok{: 1.17 }\OtherTok{$Date}\NormalTok{: }
            
\KeywordTok{FC}\NormalTok{:/}\KeywordTok{>} \NormalTok{add /gridpp/user/a/ada.lovelace/TODO.md TODO.md UKI-LT2-QMUL2-disk}
\KeywordTok{File} \NormalTok{/gridpp/user/a/ada.lovelace/TODO.md successfully uploaded...}
\end{Highlighting}
\end{Shaded}

We can now set the \texttt{owner} index for the LFN using the
\texttt{meta\ set} command:

\begin{Shaded}
\begin{Highlighting}[]
\KeywordTok{FC}\NormalTok{:/}\KeywordTok{>} \NormalTok{meta set /gridpp/user/a/ada.lovelace/TODO.md owner ada.lovelace}
\KeywordTok{/gridpp/user/a/ada.lovelace/TODO.md} \NormalTok{owner ada.lovelace}
\end{Highlighting}
\end{Shaded}

We can now find the file again using the \texttt{find} command:

\begin{Shaded}
\begin{Highlighting}[]
\KeywordTok{FC}\NormalTok{:/}\KeywordTok{>} \NormalTok{find / owner=ada.lovelace}
\KeywordTok{Query}\NormalTok{: \{}\StringTok{'owner'}\NormalTok{: }\StringTok{'ada.lovelace'}\NormalTok{\}}
\KeywordTok{/gridpp/user/a/ada.lovelace/TODO.md}
\KeywordTok{QueryTime} \NormalTok{0.01 sec}
\end{Highlighting}
\end{Shaded}

As we've said before, the DFC CLI is useful for small-scale operations
on your data. Hopefully, though, you can start to appreciate the power
of \textbf{metadata} when it comes to organising your data and
performing analyses on it.

The most important thing for the moment, though, is that we can now put
data on the Grid (i.e.~on a Storage Element). This means we can use it
in our Grid jobs without needing to upload with our job as an
\texttt{inputfile}. We'll now complete making our example workflow fully
Grid-enabled in the next section.


%=============================================================================
\subsection{Checklist}
\label{putting-data-on-the-grid---checklist}
%=============================================================================

\begin{itemize}
\tightlist
\item
  I know what a Grid Storage Element (SE) is;
\item
  I know what the DIRAC File Catalog (DFC) is and what it is used for;
\item
  I know what LFN stands for and what it means with respect to the DFC;
\item
  I can access the DIRAC File Catalog Command Line Interface (DFC CLI);
\item
  I can find the Grid Storage Elements (SEs) available for me to use;
\item
  I can use tab-complete with \texttt{dirac-dms-} to find the available
  DIRAC command;
\item
  I can list the contents of my Virtual Organisations's area in the DFC
\item
  I can create a user area within this area and set the permissions
  accordingly;
\item
  Using both the DFC CLI and the DFC command line tools, I can:
\item
  upload and download files to and from an SE;
\item
  replicate files to another SE;
\item
  remove a replica from a specified SE;
\item
  remove a file from the DFC.
\item
  I know how to assign metadata to an LFN using the DFC CLI.
\end{itemize}


%=============================================================================
\subsection{Testing}
\label{putting-data-on-the-grid---testing}
%=============================================================================

\begin{itemize}
\tightlist
\item
  \textbf{Accessing data on a GridPP Storage Element (SE)}: If
  everything is setup correctly, the following commands should result in
  the output below:
\end{itemize}

\begin{Shaded}
\begin{Highlighting}[]
\NormalTok{$ }\KeywordTok{cd} \NormalTok{~}
\NormalTok{$ }\KeywordTok{mkdir} \NormalTok{tmp}
\NormalTok{$ }\KeywordTok{cd} \NormalTok{tmp}
\NormalTok{$ }\KeywordTok{dirac-dms-get-file} \NormalTok{LFN:/gridpp/userguide/WELCOME.md}
\NormalTok{\{}\StringTok{'Failed'}\NormalTok{: }\DataTypeTok{\{\}}\NormalTok{,}
 \StringTok{'Successful'}\NormalTok{: \{}\StringTok{'/gridpp/userguide/WELCOME.md'}\NormalTok{: }\StringTok{'/home/alovelace/tmp/WELCOME.md'}\NormalTok{\}\}}
\NormalTok{$ }\KeywordTok{cat} \NormalTok{WELCOME.md}
\CommentTok{#Welcome to GridPP!}

\KeywordTok{It} \NormalTok{looks like your download has worked. Congratulations! }
\end{Highlighting}
\end{Shaded}

\emph{You should have been able to follow everything else in the
previous subsections too, of course - but much of the input and output
will depend on your username, VO, etc. - the real test come when you
start putting data in your own user area in the next section!}


