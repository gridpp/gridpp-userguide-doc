%=============================================================================
\subsection{The DIRAC Command Line Tools}
\label{the-dirac-command-line-tools}
%=============================================================================
So you've mastered the DFC Command Line Interface. Great stuff. What
you'll have probably noticed is that, while it's great for small-scale
operations, it's not ideal for doing things with lots of files on any
sort of scale. We will therefore want to take a look at the
\textbf{DIRAC command line tools} for data management.

\begin{infobox}{The command line tools}
\emph{All of the DIRAC command line tools start with}
\code{dirac-}. \emph{The data management tools start with}
\code{dirac-dms-}, \emph{as in Data Management System.
Press the tab key after typing} \texttt{dirac-dms-}
\emph{to see all of the available commands.}
\end{infobox}

Why are these commands useful? Well, it means you can use
\emph{scripting} to automate large-scale tasks involving many files.
There are many ways to \emph{script} the DIRAC (or indeed any command
line) commands. You've probably got your own preferred method that
reflects your coding background. For the purposes of the UserGuide,
we'll use simple bits of \textbf{Python} code (along with Python-based
file management libraries) to generate some simple \texttt{bash} scripts
that can then be run to perform the DIRAC operations we want to perform.

\begin{infobox}{Scripting DIRAC commands}
\emph{Of course, bash experts will be able to write scripts that perform all
of the operations below purely in bash. This is left as an exercise for
the reader - answers on a punch card please! (Also, we'll be using
Python for the DIRAC Python API, so it's not a bad thing to use Python
at this stage!)}
\end{infobox}

%-----------------------------------------------------------------------------
\subsubsection{Uploading files}
\label{uploading-files-tools}
%-----------------------------------------------------------------------------
The DIRAC file upload command takes the following form:

\begin{verbatim}
$ dirac-dms-add-file <LFN> <FILE> <SE>
\end{verbatim}

where:

\begin{itemize}
\tightlist
\item \texttt{\textless{}LFN\textgreater{}} is the Logical File Name
(LFN) of the entry for the file in the DIRAC File Catalog (DFC);
\item \texttt{\textless{}FILE\textgreater{}} is the path to the file on your
local machine, and;
\item \texttt{\textless{}SE\textgreater{}} is the name
of the destination Storage Element (SE).
\end{itemize}

\begin{infobox}{Finding SE names}
\emph{Remember, you can find the names of the available SEs with the
dirac-dms-show-se-status command.}
\end{infobox}

Suppose we have a number of files on our local machine in
\texttt{/home/gridpp/mydata/} that we want to upload to the grid. The
following Python code will generate a \texttt{bash} script that will
upload them to one of the Queen Mary Storage Elements:

\begin{Shaded}
\begin{Highlighting}[]
\NormalTok{$ cat make_upload_script.py}
\CommentTok{#!/usr/bin/env python}
\CommentTok{# -*- coding: utf-8 -*-}

\ImportTok{import} \NormalTok{os, glob}

\NormalTok{data_path }\OperatorTok{=} \StringTok{'/home/gridpp/mydata'}

\NormalTok{lfn_dir }\OperatorTok{=} \StringTok{'/gridpp/user/a/ada.lovelace/mydata/'}

\NormalTok{se }\OperatorTok{=} \StringTok{'UKI-LT2-QMUL2-disk'}

\NormalTok{s }\OperatorTok{=} \StringTok{"#!/bin/bash}\CharTok{\textbackslash{}n}\StringTok{"}

\ControlFlowTok{for} \NormalTok{my_file }\OperatorTok{in} \BuiltInTok{sorted}\NormalTok{(glob.glob(data_path }\OperatorTok{+} \StringTok{"/*"}\NormalTok{)):}
    \NormalTok{base_name  }\OperatorTok{=} \NormalTok{os.path.basename(my_file)}
    \NormalTok{upload_lfn }\OperatorTok{=} \NormalTok{os.path.join(lfn_dir, base_name)}
    \NormalTok{s }\OperatorTok{+=} \StringTok{"dirac-dms-add-file }\SpecialCharTok{%s}\StringTok{ }\SpecialCharTok{%s}\StringTok{ }\SpecialCharTok \NormalTok{(upload_lfn, my_file, se)}

\ControlFlowTok{with} \BuiltInTok{open}\NormalTok{(}\StringTok{"upload_script.sh"}\NormalTok{, }\StringTok{"w"}\NormalTok{) }\ImportTok{as} \NormalTok{sf:}
    \NormalTok{sf.write(s)}
\end{Highlighting}
\end{Shaded}

After you've generated a proxy and sourced the DIRAC environment, you
can run the generated script as follows:

\begin{Shaded}
\begin{Highlighting}[]
\NormalTok{$ }\KeywordTok{python} \NormalTok{make_upload_script.py}
\NormalTok{$ }\KeywordTok{chmod} \NormalTok{a+x upload_script.sh}
\NormalTok{$ }\KeywordTok{.} \KeywordTok{upload_script.sh}
\end{Highlighting}
\end{Shaded}

The results of this will, of course, depend on the contents of
\texttt{/home/gridpp/mydata/}, but all being well you should see the
message:

\begin{verbatim}
Successfully uploaded file to UKI-LT2-QMUL2-disk
\end{verbatim}

(or whichever SE you specified in your Python code) after each file has
been uploaded.

\begin{infobox}{Using Screen}
\emph{If you're uploading a lot of files, you may wish to consider using
something like the screen tool so that you can log off your terminal
session and come back to it later.}
\end{infobox}

And there we go! Multiple file uploads, all registered in the DIRAC File
Catalog, using a DIRAC command line tool and a bit of (admitedly
slightly clumsy) coding.

%-----------------------------------------------------------------------------
\subsubsection{Replicating files}
\label{replicating-files-tools}
%-----------------------------------------------------------------------------
Now, as we did with the DFC CLI, we can also make replicas of files,
list information about the replicas of a given file, and remove replicas
with the following command line tools:

\begin{verbatim}
dirac-dms-replicate-lfn <LFN> <SE>
dirac-dms-lfn-replicas <LFN>
dirac-dms-remove-replicas <LFN> <SE>
\end{verbatim}

Likewise, we can take the same approach with\ldots{}

%-----------------------------------------------------------------------------
\subsubsection{Downloading and removing files}
\label{downloading-and-removing-files-tools}
%-----------------------------------------------------------------------------

\begin{verbatim}
dirac-dms-get-file <LFN>
dirac-dms-remove-files <LFN>
\end{verbatim}

i.e.~the DIRAC command line tools exist for these operations. However,
getting information from the DFC about which files you would like to
replicate, download, remove, etc. is non-trivial when taking the command
line approach. This is especially true if you're writing scripts.

One approach is to use the \textbf{metadata} functionality the DIRAC
File Catalog provides to find files of interest.

\begin{infobox}{Metadata}
\term{Metadata} \emph{is data about the data. By assigning metadata to the files we
upload to the DIRAC File Catalog, we can perform queries that will
select only the files we are interested in. It also helps us to manage
our data. We'll find out more about the DFC's metadata functionality
later.}
\end{infobox}

The \texttt{dirac-dms-find-lfns} command finds LFNs based on the DFC
path and metadata query supplied as options. For example, to find all
files in the DFC that have been assigned to the experiment
\texttt{UserGuide}, we can type:

\begin{Shaded}
\begin{Highlighting}[]
\KeywordTok{dirac-dms-find-lfns} \NormalTok{Path=/ }\StringTok{"experiment=UserGuide"}
\NormalTok{\{}\StringTok{'experiment'}\NormalTok{: }\StringTok{'UserGuide'}\NormalTok{\}}
\KeywordTok{/gridpp/userguide/WELCOME.md}
\end{Highlighting}
\end{Shaded}

\texttt{experiment} here is the \textbf{metadata element} or
\textbf{index}. This is a string assigned to the file's LFN that, in
this case, has the value \texttt{UserGuide}. We can use the results of
this to download the files we want.

\begin{Shaded}
\begin{Highlighting}[]
\NormalTok{$ }\KeywordTok{dirac-dms-get-file} \NormalTok{/gridpp/userguide/WELCOME.md}
\NormalTok{\{}\StringTok{'Failed'}\NormalTok{: }\DataTypeTok{\{\}}\NormalTok{,}
 \StringTok{'Successful'}\NormalTok{: \{}\StringTok{'/gridpp/userguide/WELCOME.md'}\NormalTok{: }\StringTok{'/home/gridpp/tmp/WELCOME.md'}\NormalTok{\}\}}
\NormalTok{$ }\KeywordTok{cat} \NormalTok{WELCOME.md}
\CommentTok{#Welcome to GridPP!}

\KeywordTok{It} \NormalTok{looks like your download has worked. Congratulations!}
\end{Highlighting}
\end{Shaded}

Let's take a closer look at the DFC's metadata functionality using the
DFC CLI.
