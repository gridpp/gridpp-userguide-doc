%=============================================================================
\subsection{The DFC Command Line Interface}
\label{the-dfc-command-line-interface}
%=============================================================================
The DIRAC File Catalog (DFC) Command Line Interface (CLI), a.k.a. the
\textbf{DFC CLI}, provides a way of interacting with DIRAC's File
Catalog via - you guessed it - the command line. The DFC CLI lets you
manually upload and download files to Storage Elements (SEs), browse the
DFC associated with your Virtual Organisation (VO), create and remove
directories in the DFC, and manage the replicas associated with each
entry in the DFC.

\begin{infobox}{Using the DFC CLI}
\emph{The DFC CLI is great for small-scale tasks such as creating and tweaking
test data sets, but ultimately we will want to use scripts to help
coordinate large-scale upload operations and managing metadata
(i.e.~data about the data).}
\end{infobox}

%-----------------------------------------------------------------------------
\subsubsection{Getting started with the DFC CLI}
\label{getting-started-with-the-dfc-cli}
%-----------------------------------------------------------------------------

\term{Accessing the DFC CLI}

The DFC CLI is accessed via a DIRAC command, so we'll need to source our
DIRAC environment and generate a DIRAC proxy.

\begin{Shaded}
\begin{Highlighting}[]
\NormalTok{$ }\KeywordTok{source} \NormalTok{/cvmfs/ganga.cern.ch/dirac_ui/bashrc}
\NormalTok{$ }\KeywordTok{dirac-proxy-init} \NormalTok{-g gridpp_user -M}
\KeywordTok{Generating} \NormalTok{proxy... }
\KeywordTok{Enter} \NormalTok{Certificate password: }\CommentTok{# Enter your grid certificate password...}
\KeywordTok{.}
\KeywordTok{.} \NormalTok{[}\KeywordTok{Proxy} \NormalTok{information-based output.]}
\KeywordTok{.}
\end{Highlighting}
\end{Shaded}

\begin{warningbox}{Which VO?}
\emph{If you wish to use a different VO, replace gridpp with the name of the
VO in the commands in this section.}
\end{warningbox}

The DFC CLI is then started with the following DIRAC command:

\begin{Shaded}
\begin{Highlighting}[]
\NormalTok{$ }\KeywordTok{dirac-dms-filecatalog-cli} 
\KeywordTok{Starting} \NormalTok{FileCatalog client}

\KeywordTok{File} \NormalTok{Catalog Client }\OtherTok{$Revision}\NormalTok{: 1.17 }\OtherTok{$Date}\NormalTok{: }
            
\KeywordTok{FC}\NormalTok{:/}\KeywordTok{>}
\end{Highlighting}
\end{Shaded}

\begin{infobox}{DIRAC command groupings}
\emph{We'll come back to the DIRAC command line tools in the next section, but
the} \code{dirac-dms-} \emph{at the start of the command refers to the DIRAC Data
Management System tools. All DIRAC commands are grouped in this way
which, combined with tab completion, can be very handy for finding the
command you're looking for!}
\end{infobox}

The \texttt{FC:/\textgreater{}} at the command prompt tells you that
you're in the DFC CLI. You can now explore the DFC using commands that
are very similar to those used with a typical UNIX file system. Let's do
this now.

%-----------------------------------------------------------------------------
\subsubsection{Finding your user space in the DFC}
\label{finding-your-user-space-in-the-dfc}
%-----------------------------------------------------------------------------
Let's start by listing the root directories in the DFC, which will give
us a list of the Virtual Organisations supported by GridPP DIRAC:

\begin{Shaded}
\begin{Highlighting}[]
\KeywordTok{FC}\NormalTok{:/}\KeywordTok{>} \NormalTok{ls}
\KeywordTok{cernatschool.org}
\KeywordTok{gridpp}
\KeywordTok{vo.londongrid.ac.uk}
\KeywordTok{vo.northgrid.ac.uk}
\KeywordTok{vo.scotgrid.ac.uk}
\KeywordTok{vo.southgrid.ac.uk}
\end{Highlighting}
\end{Shaded}

We're using GridPP DIRAC as a member of \texttt{gridpp} VO, so let's
move into that directory.

\begin{Shaded}
\begin{Highlighting}[]
\KeywordTok{FC}\NormalTok{:/}\KeywordTok{>} \NormalTok{cd gridpp/user}
\end{Highlighting}
\end{Shaded}

If one hasn't been created for you already, you can create your own user
space on the VO's File Catalog like so:

\begin{Shaded}
\begin{Highlighting}[]
\KeywordTok{FC}\NormalTok{:/gridpp/user}\KeywordTok{>} \NormalTok{cd a}
\KeywordTok{FC}\NormalTok{:/gridpp/user/a}\KeywordTok{>} \NormalTok{mkdir ada.lovelace}
\KeywordTok{FC}\NormalTok{:/gridpp/user/a}\KeywordTok{>} \NormalTok{chmod 755 ada.lovelace}
\KeywordTok{FC}\NormalTok{:/gridpp/user/a}\KeywordTok{>} \NormalTok{ls -la}
\KeywordTok{drwxr-xr-x} \NormalTok{0 ada.lovelace gridpp_user 0 2015-12-16 10:24:54 ada.lovelace }
\KeywordTok{FC}\NormalTok{:/gridpp/user/a}\KeywordTok{>} \NormalTok{exit}
\end{Highlighting}
\end{Shaded}

\begin{infobox}{Your DIRAC username}
\emph{If you don't know your DIRAC username (which should be used as your user
directory), exit the DFC CLI and use the dirac-proxy-info command.}
\end{infobox}

\begin{infobox}{Listing files}
\emph{Using the} \code{-la} \emph{option with the}
\code{ls} \emph{command works just as it does with the
normal command line, allowing you to see file owners, groups (VOs),
permissions, etc.}
\end{infobox}

\begin{warningbox}{File permissions}
\emph{Don't forget to change the file permissions on your files so that other
users can't modify them.}
\end{warningbox}

You've now got your own space on the GridPP DFC. Let's put some files in
it.

%-----------------------------------------------------------------------------
\subsubsection{Uploading files}
\label{uploading-files}
%-----------------------------------------------------------------------------
Firstly, we'll need a file to upload. Any file will do, but to keep
things simple let's create one in a temporary directory:

\begin{Shaded}
\begin{Highlighting}[]
\NormalTok{$ }\KeywordTok{cd} \NormalTok{~}
\NormalTok{$ }\KeywordTok{mkdir} \NormalTok{tmp}\KeywordTok{;} \KeywordTok{cd} \NormalTok{tmp}
\NormalTok{$ }\KeywordTok{vim} \NormalTok{TEST.md }\CommentTok{# Or whichever editor you use...}
\NormalTok{$ }\KeywordTok{cat} \NormalTok{TEST.md}
\CommentTok{#Hello Grid!}
\KeywordTok{This} \NormalTok{is a test **MarkDown file**.}
\end{Highlighting}
\end{Shaded}

Next we'll need to know which \textbf{Storage Elements} are available to
us.

\begin{infobox}{Storage Elements}
\emph{Storage Elements} ``are physical sites where data are stored and
accessed, for example, physical file systems, disk caches or
hierarchical mass storage systems. Storage Elements manage storage and
enforce authorization policies on who is allowed to create, delete and
access physical files. They enforce local as well as Virtual
Organization policies for the use of storage resources. They guarantee
that physical names for data objects are valid and unique on the storage
device(s), and they provide data access. A storage element is an
interface for grid jobs and grid users to access underlying storage
through the Storage Resource Management protocol (SRM), the Globus Grid
FTP protocol, and possibly other interfaces as well.''

\emph{Credit: Open Science Grid (2012)}
\end{infobox}

We can list the available SEs with the following DIRAC command:

\begin{Shaded}
\begin{Highlighting}[]
\NormalTok{$ }\KeywordTok{dirac-dms-show-se-status} 
\KeywordTok{SE}                           \NormalTok{ReadAccess WriteAccess RemoveAccess CheckAccess }
\NormalTok{=============================================================================}
\NormalTok{[}\KeywordTok{...} \NormalTok{more disks ...]}
\KeywordTok{UKI-LT2-QMUL2-disk}           \NormalTok{Active     Active      Unknown      Unknown     }
\NormalTok{[}\KeywordTok{...} \NormalTok{more disks ...]}
\KeywordTok{UKI-NORTHGRID-LIV-HEP-disk}   \NormalTok{Active     Active      Unknown      Unknown}
\NormalTok{[}\KeywordTok{...} \NormalTok{more disks ...]}
\end{Highlighting}
\end{Shaded}

While we don't need to know the details of where and how our data will
be stored on an SE, we do need to know its name. We'll use the
\texttt{UKI-LT2-QMUL2-disk} SE for now. We add the file to the DFC as
follows using the \texttt{add} command, which takes the following
arguments:

\begin{verbatim}
add <LFN> <Local file name> <SE name>
\end{verbatim}

where:

\begin{itemize}
\tightlist
\item
  \texttt{\textless{}LFN\textgreater{}} is the \textbf{Logical File
  Name} (LFN) of the file in the DFC. This can either be relative to
  your current position in the DFC (which can be found with the
  \texttt{pwd} command in the DFC CLI), or made absolute by preceeding
  the name with a slash \texttt{/};
\item
  \texttt{\textless{}Local\ file\ name\textgreater{}} should be the name
  of the local file you want to upload. Again, this can be relative to
  wherever you were on your local system when you started the DFC CLI,
  or the absolute path to the file on your local system;
\item
  \texttt{\textless{}SE\ name\textgreater{}} is the name of the SE as
  retrived from the \texttt{dirac-dms-show-se-status} command.
\end{itemize}

Let's add our file to the grid now.

\begin{Shaded}
\begin{Highlighting}[]
\NormalTok{$ }\KeywordTok{dirac-dms-filecatalog-cli}
\KeywordTok{Starting} \NormalTok{FileCatalog client}

\KeywordTok{File} \NormalTok{Catalog Client }\OtherTok{$Revision}\NormalTok{: 1.17 }\OtherTok{$Date}\NormalTok{: }
            
\KeywordTok{FC}\NormalTok{:/}\KeywordTok{>} \NormalTok{cd /gridpp/user/a/ada.lovelace}
\KeywordTok{FC}\NormalTok{:/gridpp/user/a/ada.lovelace}\KeywordTok{>} \NormalTok{mkdir tmp}
\KeywordTok{FC}\NormalTok{:/gridpp/user/a/ada.lovelace}\KeywordTok{>} \NormalTok{cd tmp}
\KeywordTok{FC}\NormalTok{:/gridpp/user/a/ada.lovelace}\KeywordTok{>} \NormalTok{add TEST.md TEST.md UKI-LT2-QMUL2-disk}

\KeywordTok{File} \NormalTok{/gridpp/user/a/ada.lovelace/tmp/TEST.md successfully uploaded...}
\KeywordTok{FC}\NormalTok{:/gridpp/user/a/ada.lovelace/tmp}\KeywordTok{>}\NormalTok{ls -la}
\KeywordTok{-rwxrwxr-x} \NormalTok{1 ada.lovelace gridpp_user 47 2015-12-16 11:47:28 TEST.md}
\end{Highlighting}
\end{Shaded}

And there we go! Your first file has been uploaded to a Storage Element
on the grid. Have a biscuit. You've earned it.

%-----------------------------------------------------------------------------
\subsubsection{Replicating files}
\label{replicating-files}
%-----------------------------------------------------------------------------
Part of the joy of using the grid is being able to distribute
computational tasks to different sites. However, if you want to look at
the same data with a different task at different sites in an efficient
manner, ideally you'd need copies of that data at those sites. This
strategy also makes sense from a backup/redundancy perspective. We can
achieve this on the grid by using \emph{replicas}.

\begin{infobox}{Replicas}
\emph{A replica is a copy of a given file that is located on a different
Storage Element (SE). The file is identified by its Logical File Name
(LFN) in the DIRAC File Catalog (DFC). Associated with each LFN entry is
a list of SEs where replicas of the file can be found.}
\end{infobox}

To list the locations of replicas for a given file catalog entry, we use
the \texttt{replicas} command in the DFC CLI:

\begin{verbatim}
replicas <LFN>
\end{verbatim}

so continuing with our example:

\begin{Shaded}
\begin{Highlighting}[]
\KeywordTok{FC}\NormalTok{:/gridpp/user/a/ada.lovelace/tmp}\KeywordTok{>}\NormalTok{replicas TEST.md}
\KeywordTok{lfn}\NormalTok{: /gridpp/user/a/ada.lovelace/tmp/TEST.md}
\end{Highlighting}
\end{Shaded}

We replicate files with the \texttt{replicate} command:

\begin{verbatim}
replicate <LFN> <SE name>
\end{verbatim}

Let's replicate our test file to the Liverpool disk and check that the
replica list has been updated:

\begin{Shaded}
\begin{Highlighting}[]
\KeywordTok{FC}\NormalTok{:/gridpp/user/a/ada.lovelace/tmp}\KeywordTok{>}\NormalTok{replicate TEST.md UKI-NORTHGRID-LIV-HEP-disk}
\end{Highlighting}
\end{Shaded}

Replicas can be removed with the \texttt{rmreplica} command:

\begin{verbatim}
rmreplica <LFN> <SE name>
\end{verbatim}

Let's remove the Liverpool disk replica:

\begin{Shaded}
\begin{Highlighting}[]
\KeywordTok{FC}\NormalTok{:/gridpp/user/a/ada.lovelace/tmp}\KeywordTok{>}\NormalTok{rmreplica TEST.md UKI-NORTHGRID-LIV-HEP-disk}
\KeywordTok{lfn}\NormalTok{: /gridpp/user/a/ada.lovelace/tmp/TEST.md}
\KeywordTok{Replica} \NormalTok{at UKI-NORTHGRID-LIV-HEP-disk moved to Trash Bin}
\end{Highlighting}
\end{Shaded}

Finally, we can remove a file completely using the (somewhat familiar)
\texttt{rm} command:

\begin{verbatim}
rm <LFN>
\end{verbatim}

Let's tidy up our test file:

\begin{Shaded}
\begin{Highlighting}[]
\KeywordTok{FC}\NormalTok{:/gridpp/user/a/ada.lovelace/tmp}\KeywordTok{>}\NormalTok{rm TEST.md}
\KeywordTok{lfn}\NormalTok{: /gridpp/user/a/ada.lovelace/tmp/TEST.md}
\KeywordTok{File} \NormalTok{/gridpp/user/a/ada.lovelace/tmp/TEST.md removed from the catalog}
\end{Highlighting}
\end{Shaded}

%-----------------------------------------------------------------------------
\subsubsection{Downloading files}
\label{downloading-files}
%-----------------------------------------------------------------------------
Finally, we can download files using the DFC CLI with the \texttt{get}
command:

\begin{verbatim}
get <LFN> [<local directory>]
\end{verbatim}

Note that the local directory argument is optional. Let's download a
test file from the \texttt{gridpp} examples directory now:

\begin{Shaded}
\begin{Highlighting}[]
\KeywordTok{FC}\NormalTok{:/}\KeywordTok{>} \NormalTok{get /gridpp/userguide/WELCOME.md ./.}
\KeywordTok{FC}\NormalTok{:/}\KeywordTok{>} \NormalTok{exit}
\NormalTok{$ }\KeywordTok{cat} \NormalTok{WELCOME.md}
\CommentTok{#Welcome to GridPP!}

\KeywordTok{It} \NormalTok{looks like your download has worked. Congratulations!}
\NormalTok{$ }\KeywordTok{rm} \NormalTok{WELCOME.md}
\end{Highlighting}
\end{Shaded}

As we said earlier, the DFC CLI is only useful for small-scale
operations. On our way to scaling up, we can look at starting to
automate our workflows using scripts. In the next section we'll look at
how the DIRAC command line tools can help with this.
