%%%%%%%%%%%%%%%%%%%%%%%%%%%%%%%%%%%%%%%%%%%%%%%%%%%%%%%%%%%%%%%%%%%%%%%%%%%%%%%
\section{Introduction}
\label{sec:intro}
%%%%%%%%%%%%%%%%%%%%%%%%%%%%%%%%%%%%%%%%%%%%%%%%%%%%%%%%%%%%%%%%%%%%%%%%%%%%%%%

Welcome to the \emph{GridPP UserGuide}. The
\href{https://www.gridpp.ac.uk}{GridPP Collaboration}~\cite{gridpp2006,gridpp2009}
is a community of
particle physicists and computer scientists based in the United Kingdom
and at \href{http://cern.ch}{CERN}. It supports tens of thousands of CPU
cores and petabytes of data storage across the UK which, amongst other
things, played a crucial role in the discovery of the
Higgs boson~\cite{CMS2012a,ATLAS2012c}
at
\href{http://cern.ch}{CERN}'s Large Hadron Collider~\cite{LHC2008}.
The aim of this
document is to help new users - like you - join this community and
access these resources to make a difference to the world beyond the
realm of particle physics.

So, if you have a data-intensive problem that could be solved using
large-scale distributed computing, read on!

%=============================================================================
\subsection{Who is this guide for?}
\label{who-is-this-guide-for}
%=============================================================================
This guide is primarily aimed at people from user communities that have
not previously engaged with grid (a.k.a. distributed computing)
technology. You could be:

\begin{itemize}
\tightlist
\item
  a researcher from a UK institution with a problem that could be solved
  by the application of thousands of computers running software in
  parallel over large, structured data sets;
\item
  a student from the UK who would benefit from being able to access
  computing and data storage resources that your own institution cannot
  provide (i.e.~your school);
\item
  a tech entrepreneur from a start-up or Small-to-Medium Enterprise
  (SME) who would like to test how your software or app scales to
  thousands of machines for zero cost and minimal risk.
\end{itemize}

The \emph{GridPP UserGuide} isn't really aimed at:

\begin{itemize}
\tightlist
\item
  Members of scientific collaborations who already have a grid presence
  and infrastructure in place (e.g.~CMS, ATLAS, SNO++, T2K);
\item
  Users from outside of the UK - you should refer to your own country's
  National Grid Initiative (NGI) to find out about the best way of
  getting on the grid,
\end{itemize}

although it may serve as a useful reference for members of these
communities.

%_____________________________________________________________________________
\begin{warningbox}{Assumed knowledge}
\emph{While every effort has been made to cover as many bases as possible,
some computing knowledge is assumed. You can read more about what you
might need to know in the prerequisites section.}
\end{warningbox}
%_____________________________________________________________________________


%=============================================================================
\subsection{What do I do next?}
\label{what-do-i-do-next}
%=============================================================================
You should read \emph{Before We Begin} (Section~\ref{sec:bwb})
%\href{before-we-begin/before-we-begin.html}{Before We Begin}
to go over the \emph{Prerequisites} (Section~\ref{sec:prerequisites}),
%\href{before-we-begin/prerequisites.html}{prerequisites},
the \emph{conventions} (Section~\ref{sec:conventions}),
%\href{before-we-begin/conventions.html}{UserGuide conventions},
and how to \emph{get help} from the GridPP community (Section~\ref{sec:help}).
%\href{before-we-begin/getting-help.html}{help and support}
If you do not have access to a \textbf{Grid User
Interface} (UI) (or don't know what that means), you should then look at
creating one following the instructions
in Appendix~\ref{sec:gridppcernvm}.
%\href{gridpp-cernvm/gridpp-cernvm.html}{here}.
Then it's simply a case
of getting a \textbf{grid certificate}, joining a \textbf{Virtual
Organisation} (VO), and getting on the Grid!

%_____________________________________________________________________________
\begin{hintbox}{What does this all mean?}
\emph{Don't worry, we'll explain all of those terms as we go along - mostly
using little information boxes like this one. For now, though, you can
start here.}
\end{hintbox}
%_____________________________________________________________________________
